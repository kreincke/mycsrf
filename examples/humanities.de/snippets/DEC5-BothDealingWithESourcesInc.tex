%fodina humanitied 'for being included' snippet template
%
% (c) Karsten Reincke, Frankfurt a.M. 2010, 2011, ff.
%
% This LaTeX-File is licensed under the Creative Commons Attribution-ShareAlike
% 3.0 Germany License (http://creativecommons.org/licenses/by-sa/3.0/de/): Feel
% free 'to share (to copy, distribute and transmit)' or 'to remix (to adapt)'
% it, if you '... distribute the resulting work under the same or similar
% license to this one' and if you respect how 'you must attribute the work in
% the manner specified by the author ...':
%
% In an internet based reuse please link the reused parts to www.fodina.de and
% mention the original author Karsten Reincke in a suitable manner. In a
% paper-like reuse please insert a short hint to www.fodina.de and to the
% original author, Karsten Reincke, into your preface. For normal quotations
% please use the scientific standard to cite.
%
%% use all entries of the bibliography
%\nocite{*}
\section{Electronic Resources - oder: das Problem volatiler Quellen}

\emph{mycsrf} folgt von der Form her einem traditionellen, manche mögen sagen:
veralteten Ideal des Zitierens. Der Zweck dessen ist aber sehr mordern: es gilt
zu gewährleisten, dass die Quellen für delegierte Beweise leicht zu überprüfen
sind. Mit diesem Wunsch steht \emph{mycsrf} nicht allein!

Ein gutes Beispiel mit einem ähnlich ausgerichteten Anspruch ist das 2013
in der 16. Auflage erschienene \enquote{Standardwerk} zum Thema
\enquote{Wissenschaftliches
Arbeiten}\footnote{\cite[vgl.][\nopage]{Theisen2013a}. Das Wort
\enquote{Standardwerk} findet sich auf dem Einband. Für sich genommen ist es
natürlich 'nur'  Werbung. Allerdings ist das Buch zwischen 1984 und 2013 in 16
Auflagen erschienen. Das unterstreicht seine Bedeutung von den Fakten her.}.
Es verweist - unter Rückgriff auf andere Quellen - zunächst auf die gesetzlichen
Grundlagen für die Notwendigkeit des korrekten Zitierens, insbesondere auf das
(deutsche) Urhebergesetz\footcite[vgl.][159]{Theisen2013a}. Den Sinn des
Zitierens aus forschungslogischer Sicht verknüpft es dann mit dem Kriterium der
\enquote{Zitierfähigkeit}: die \enquote{Erfordernis}, will sagen:
Voraussetzung, dass \enquote{alle Quellen und Sekundärmaterialien} einer
wissenschaftlichen Arbeit \enquote{[\ldots] in irgendeiner Form [\ldots]
veröffentlicht worden (seien) [\ldots] (stelle sicher), dass für
wissenschaftliche Zwecke nur solches Material verwendet (werde), das
nachvollziehbar und damit auch \emph{kontrollierbar} (sei)}\footcite[vgl.][160;
herv. i.O]{Theisen2013a}.

Um diese 'Zugänglichmachung' zwecks Überprüfbarkeit gewährleisten zu können, hat
sich über die Jahrhunderte ein arbeitsteiliges Modell entwickelt: Jedes
deutschsprachige Buch muss bei der deutschen Nationalbibliothek hinterlegt sein,
die Aufnahme in die Nationalbibliografie beruht dann - wie es heißt - auf einer
\enquote{\enquote{Autopsie} des eingereichten
Buches}\footcite[vgl.][68]{Theisen2013a}. Für andere Länder existieren ähnliche
Gewährleistungssysteme. Wenigstens über die nationalen Bibliotheken hinweg gibt
es zudem eine Kooperation: sie zusammen erheben den Anspruch, die wichtigste
auch nicht deutsche Fachliteratur abzudecken und den Forschern -- nötigenfalls
im Austauschverfahren -- zur Verfügung zu stellen.

Damit kann die Forschungsgemeinschaft zwei Aspekte voraussetzen, die zusammen
eine dauerhafte, oder wenigstens: sehr langfristige Überprüfbarkeit etablieren:
\begin{itemize}
  \item Jedes (gedruckte) Werk kann anhand genauer Angaben in einem mehr oder
  minder aufwändigen Verfahren über Bibliotheken beschafft werden.
  \item  Keine (gedrucktes) Werk kann sich 'plötzlich' ändern, sodass eine
  korrekte Seitenzahl konstant und immer wieder auf die intendierte Belegstelle
  verweist.
\end{itemize}

Unglücklicherweise gelten diese beiden Voraussetzung bei \emph{E-Books} oder
\emph{E-Papers}\footnote{Solange Fremdwörter noch nicht etabliert sind, ist es
stilistisch fraglich, ob und in welcher Form sie seriöserweise verwendet werden
dürfen. Der Duden normiert bereits die Schreibweise des Wortes \emph{E-Mail}
(\cite[vgl.][392]{Duden2009a}) Wir übertragen diese Normierung auf die Wörter
\emph{E-Book} und \emph{E-Paper}} nicht mit derselben
Verlässlichkeit\footnote{Für bestimmte Formate ergibt sich das schon aus ihrer
technischen Definition: So erlaubt etwa das ePub-Format dem Interpreter, will
sagen: dem E-Reader, das Dokument entsprechend einer personalisierten
Schriftgröße zu rendern. Bei konstanter Größe der Sichtfläche ohne horizontale
Scrollmöglichkeit führt das notwendig zu einer veränderten Umbruch und also zu
einer geänderten Seitenzählung. Der gern genutzte Kindle ist ein gutes Beispiel
dafür.}:

Trotzdem bieten Universitätsbibliotheken bestimmte Literatur 'nur' noch als
E-Books oder als E-Papers an. Dies ist ökonomisch sinnvoll. Es reduziert Lager-
und Verwaltungskosten. Allerdings 'kaufen' die Bibliotheken dabei oft nicht die
elektronischen Kopien an sich ein, um diese von ihrem eigenen Server aus an die
aus ihrer Sicht berechtigten Leser zu distribuieren. Vielmehr erwerben sie von
den Verlagen 'nur' das Recht, dass die über die Bibliothek authentifizierten
Leser sich eine Kopie aus dem Verlagsnetz downloaden dürfen\footnote{ Das ist im
Übrigen ein technischer Grund dafür, warum ein solches Downloaden seitens der
Verlage 'nur' aus dem Universtätsbibliotheksnetz heraus ermöglicht wird. Diese
Hintergründe sind mir bei verschiedenen Veranstaltung zur Bibliotheksnutzung in
Frankfurt und Darmstadt bestätitgt worden. Eine zitierfähige Beschreibung des
Verfahrens steht -- laut Auskunft der Bibliothekare (erfragt zuletzt am
26.02.2016) -- jedenfalls für Frankfurt nicht so einfach zur Verfügung.}.

Damit wird die 'Verlässlichkeitskette' beschädigt: Verlage können nun einfach
und 'unerwähnt' unter demselben Downloadlink die modifizierte Version eines
Werkes anbieten. Textstellen könnten damit geändert, Seitenzahlen verschoben
worden sein. Letztlich muss das nicht einmal absichtlich geschehen.
Schon simple technische Fehler oder Irritationen wären in der Lage, solche
Veränderungen zu bewirken.

Dieses Problem wird dadurch erschwert, dass man den zitierten E-Werken
nachträgliche Eingriffe nicht ansieht - ganz im Gegensatz zu physischen
gedruckten Werken. Diese können beschädigt werden, verbrennen oder sich sonst
wie auflösen. Allerdings sähe man ihnen dan den Verlust ihrer 'Belegkraft'
direkt an. Können sie trotzdem noch berechtigt als Zeugen fungieren, haben sie
auch denselben Inhalt, wie das unbeschädigte 'Original'. Bei E-Books und
E-Papers ist das anders. Ihnen sieht man Veränderungen und Eingriffe aus sich
heraus nicht an.

So erschüttert die Nutzung von E-Quellen die Reproduzierbarkeit; sie untergräbt
die für die Wissenschaftlichkeit konstitutive Verlässlichkeit der
Forschungsliteratur: Dass E-Books und E-Papers 'stillschweigend' ersetzt und
ihnen Änderungen nicht unbedingt angesehen werden können, macht es angreifbar,
sie zu zitieren. Denn auch 'korrekt' ausgewiesene Zitate sind dann nicht mehr in
einem letzten Sinne 'reproduzierbar' und also überprüfbar\footnote{In diesem
Zusammenhang wird erkennbar, dass Bibliotheken als staatliche Einrichtungen mit
ihrer Aufgabe, den Bestand und die Beständigkeit der Forschungsliteratur zu
horten, auch heute noch eine wissenschaftskonstitutive Funktion wahrnehmen.}.

Noch viel deutlicher tritt dieser systematische Makel in den Fällen zu Tage, wo
nicht einmal mehr Verlage hinter den über das Internet distribuierten E-Books
stehen: Das \enquote{größte Risiko bei der Verwendung von im Internet
generierten [\ldots] Materialien} liege - nach Ansicht des genannten
Standardwerks - darin, dass \enquote{[\ldots] nicht jeder Dateneingeber [\ldots]
sich bzw. seine Dokumente nachhaltig vor Manipulationen schützen
(könne)}\footcite[vgl.][85]{Theisen2013a}: \enquote{die Offenheit des
Internet-Systems (erlaube) es, Nachrichten und \emph{Daten} zu verändern oder
ganz \emph{zu verfälschen}}. Das bedeutet, dass das Internet der
\enquote{Flüchtigkeit des Mediums} und der Volatilität der URLs wegen
\enquote{[...] nur im Ausnahmefall eine Nachprüfung der Informationen über einen
längeren Zeitraum (zulasse)}\footcite[vgl.][86f; herv.i.O]{Theisen2013a}.

Und dennoch muss dieses Dilemma gelöst werden\footnote{Glücklicherweise existieren
bereits einige praktische Hinweise für einen guten Umgang mit 'Internetquellen':
So enthält etwa das \emph{MLA Handbook for Writers of Research Papers} ein
ganzes Kapitel zum \emph{Zitieren von Webpublikationen}. Seine Botschaft läuft
im Kern darauf hinaus, dass man das besondere Format der Quelle durch ein Kürzel
'Web' in den bibliographischen Daten explizit macht, dass man die URL des
zitierten Dokumentes hinzufügt und dass man auch sein je spezifisches Abrufdatum
in die bibliographischen Angaben integriert (\cite[vgl.][182 et
passim]{MlaHdb2009a}). Das hier schon mehrfach zitierte deutschsprachige Werk
geht noch einen Schritt weiter. Nachdem es die auf längere Zeit gesehen nur
eingeschränkte Nachprüfbarkeit von Internetzitaten hervorgehoben hat,
konstatiert es, dass \enquote{[...] elektronische Daten [\ldots] nachhaltig
nachgewiesen werden (müssen), so dass der Leser (oder Prüfer) sie auch zu jedem
späteren Zeitpunkt \emph{nachvollziehen} kann} (\cite[vgl.][86f]{Theisen2013a}).
Als Möglichkeiten für eine solche Verstetigung wird dann auf die
\enquote{'Screenshot'-Technik} verwiesen (\cite[vgl.][80 u. 87]{Theisen2013a}).
Die vom Autor dieser Arbeit verwendete Lösung basiert auf einer verfeinerten
Mixtur beider Vorschläge.}. Denn die E-Werke erfüllen ohne Frage das
Kriterium der Zitierfähgikeit, sofern eben \enquote{[\ldots] alle Quellen und
Sekundärmaterialien (zitierfähgig sind), die \emph{in irgendeiner Form} [\ldots]
\emph{veröffentlicht} worden sind}\footcite[vgl.][160; herv.
K.R]{Theisen2013a}. Sie zu ignorieren, ist mithin keine Option.

Um einen potentiellen Mangel an Überprüfbarkeit transparent zu machen und
möglichst auszugleichen, empfiehlt \emph{mycsrf} folgendes Verfahren:

\begin{itemize}
  \item Bei gedruckten Werken, die ein Autor über das normale Bibliotheks- resp.
  Verlagssystem beschafft und im eigentlichen Sinne des Wortes eigenhändig
  ausgewertet hat, mögen die bibliographischen Angaben im
  Literaturverzeichnis\footnote{unter Ausnutzung des BibTex Tokens
  \texttt{note}} mit dem Schlagwort \emph{Print} markiert werden. Damit
  beansprucht der Verfasser nicht nur, das Werk selbst zitiert, will sagen:
  physisch eingesehen zu haben. Er behauptet außerdem, die Angaben zur Quelle so
  genau spezifiziert zu haben, dass die Beschaffung über das normale
  Bibliothekssystem im Sinne der erwähnten Arbeitsteilung reproduzierbar sein
  sollte.
  \item Werke, die ein Autor über ein Netz in elektronischer Form eingesehen und
  ausgewertet hat, mögen im Literaturverzeichnis\footnote{unter Ausnutzung des
  BibTex Tokens \texttt{note}}  -- entsprechend des Formates -- mit
  \emph{[BibWeb$|$FreeWeb]/[PDF$|$HTML$|$\ldots]} markiert werden. Dabei stehe
  \emph{BibWeb} für ein durch eine Universitätsbibliothek bereitgestelltes
  Netz\footnote{das entweder physische Präsenz des Auswertenden in den
  Bibliotheksgebäuden oder die Nutzung eines entsprechenden VPN voraussetzt},
  während \emph{FreeWeb} das frei zugängliche Internet meine. Bei frei über das
  Internet zugänglichen Werken möge der Autor außerdem die URL und das Datum
  vermerken, unter der resp. an dem er die Seite, das E-Book oder das E-Paper
  eingesehen hat. Zudem sollte der Autor in beiden Fällen -- wo irgend möglich
  -- das eingesehene Werk als elektronische Kopie sichern. Diese darf er aus
  Urheberrechtsgründen natürlich nicht frei distribuieren. Für die Überprüfung
  seiner Zitate wird er sie im Einzelfall aber -- wo nötig -- gern in
  angemessener Weise zur Verfügung stellen.
  \item In sehr selten Fällen wird der Autor die bibliographischen Angaben -
  etwa ungenügender Kopien wegen - nicht verifizieren können, obwohl dieses Werk
  in der Forschungsliteratur durchgehend mit diesen Angaben zitiert wird. In
  solchen Fälle sollte der Autor das Werkfootnote{unter Ausnutzung
  des BibTex Tokens \texttt{note}} mit \emph{[BibWeb$|$FreeWeb] / REF}
  markieren.
  Nach menschlichem Ermessen wird ein solcher Fall aber nur im freien Web
  entstehen.
\end{itemize}

\emph{mycsrf} unterstützt diese Verfahren von sich aus: Wenn man die o.a.
Markierungen unter dem Token \texttt{note} in seine Bibtex-Bibliographie
aufnimmt, werden sie an angmessener Stelle in die bibliographischen Angaben
integriert. Die jabref-Konfiguration berücksicht die Aktivierung des Tokens
bereits.
%\bibliography{../bib/literature}
