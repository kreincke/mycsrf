% fodina Humanities LaTeX-/JuraBib-Reference-Learn-Document
%
% (c) Karsten Reincke, Frankfurt a.M. 2010, 2011, ff.
%
% This LaTeX-File is licensed under the Creative Commons Attribution-ShareAlike
% 3.0 Germany License (http://creativecommons.org/licenses/by-sa/3.0/de/): Feel
% free 'to share (to copy, distribute and transmit)' or 'to remix (to adapt)'
% it, if you '... distribute the resulting work under the same or similar
% license to this one' and if you respect how 'you must attribute the work in
% the manner specified by the author ...':
%
% In an internet based reuse please link the reused parts to www.fodina.de and
% mention the original author Karsten Reincke in a suitable manner. In a
% paper-like reuse please insert a short hint to www.fodina.de and to the
% original author, Karsten Reincke, into your preface. For normal quotations
% please use the scientific standard to cite.
%
% [ Derived from 'mykeds Classical Scholar Research Framework' 
%   mykeds-CSR-framework (c) K. Reincke CC BY 3.0  http://www.mykeds.net/ ]
%
% select the document class
% S.26: [ 10pt|11pt|12pt onecolumn|twocolumn oneside|twoside notitlepage|titlepage final|draft
%         leqno fleqn openbib a4paper|a5paper|b5paper|letterpaper|legalpaper|executivepaper openrigth ]
% S.25: { article|report|book|letter ... }
%
% oder koma-skript S.10 + 16
\documentclass[
  DIV=calc,
  BCOR=5mm,
  11pt,
  smallheadings,
  oneside,
  abstract=true,
  toc=bib,
  ngerman,english]{scrartcl}

%%% (1) general configurations %%%
\usepackage[utf8]{inputenc}

%%% (2) language specific configurations %%%
\usepackage[]{a4,babel}
\selectlanguage{english}

% package for improving the grey value and the line feed handling
\usepackage{microtype}

%language specific quoting signs
%default for language english is american style of quotes
%\usepackage[english=british]{csquotes}
\usepackage[english=american]{csquotes}

% jurabib configuration
\usepackage[see]{jurabib}
\bibliographystyle{jurabib}
\input{btexmat/fodinaHumanitiesJbibCfgEnInc}

% language specific hyphenation
\input{btexmat/fodinaHumanitiesHyphenationInc}

%%% (3) layout page configuration %%%

% select the visible parts of a page
% S.31: { plain|empty|headings|myheadings }
%\pagestyle{myheadings}
\pagestyle{headings}

% select the wished style of page-numbering
% S.32: { arabic,roman,Roman,alph,Alph }
\pagenumbering{arabic}
\setcounter{page}{1}

% select the wished distances using the general setlength order:
% S.34 { baselineskip| parskip | parindent }
% - general no indent for paragraphs
\setlength{\parindent}{0pt}
\setlength{\parskip}{1.2ex plus 0.2ex minus 0.2ex}


%%% (4) general package activation %%%
%\usepackage{utopia}
%\usepackage{courier}
%\usepackage{avant}
\usepackage[dvips]{epsfig}

% graphic
\usepackage{graphicx,color}
\usepackage{array}
\usepackage{shadow}
\usepackage{fancybox}

\usepackage{tikz}
\usetikzlibrary{arrows}
\usetikzlibrary{shapes,snakes}
\usetikzlibrary{positioning}
\usetikzlibrary{decorations.text}
\usetikzlibrary{trees}
\usetikzlibrary{matrix}

\usepackage{amsmath}
\usepackage{amsfonts}
\usepackage{amssymb}
\usepackage{wasysym}
\usepackage{chngcntr}

%- start(footnote-configuration)

% formatting the footnote with koma script tools

\deffootnote[1.5em]{1.5em}{1.5em}{\textsuperscript{\thefootnotemark)\ }}

% if document class: count footnotes from start to end
%\counterwithout{footnote}{chapter}


% Let all notes being marked with \endnote instead of \footnote
% become endnotes. This set of endnotes replaces the next 
% arising command \theendnotes - even if it is not located
% at the end of the text.

\usepackage{endnotes}

% Format endnotes as Block with indention - Solution 1
%\renewcommand\enoteformat{%
%   \noindent\theenmark.) \ \hangindent .7\parindent%
%}

% Format endnotes as Block with indention - Solution 2
\makeatletter
\def\enoteformat{\rightskip\z@ \leftskip0em \parindent=0em \parskip=0em
\leavevmode\llap{\hbox{\@theenmark.~}}}
\makeatother

\renewcommand\notesname{Annotations}
% additionally we shall active a special jurabib option
% if we want to get all jurabib footnotes as endnotes
\jurabibsetup{citetoend=true}
%- end(footnote-configuration)

% package for macking tables with broken lines
\usepackage{multirow}

\usepackage{chngcntr}

%for using label as nameref
\usepackage{nameref}

\usepackage[intoc]{nomencl}
\let\abbr\nomenclature
% Deutsche Überschrift
\renewcommand{\nomname}{Periodicals, Shortcuts, and Overlapping Abbreviations}
%\renewcommand{\nomname}{Periodika, ihre Kurzformen und generelle Abkürzungen}

\setlength{\nomlabelwidth}{.20\hsize}
\renewcommand{\nomlabel}[1]{#1 \dotfill}
% reduce the line distance
\setlength{\nomitemsep}{-\parsep}
\makenomenclature

% depth of contents
\setcounter{secnumdepth}{5}
\setcounter{tocdepth}{5}

% Hyperlinks
\usepackage{hyperref}
\hypersetup{bookmarks=true,breaklinks=true,colorlinks=true,citecolor=blue,draft=false}

\begin{document}

%% use all entries of the bliography
%\nocite{*}

%%-- start(titlepage)
\titlehead{fodina.csr.howto-\input{release}: the mykeds Classical
Scholar Research framework}
\subject{Classical Scholar Texts With \textit{jurabib}}
\title{Service For Readers and Scholars}
\subtitle{Annotations in Endnotes - but correctly}
% fodina humanities Hyphenation Module text
%
% (c) Karsten Reincke, Frankfurt a.M. 2010, 2011, ff.
%
% This LaTeX-File is licensed under the Creative Commons Attribution-ShareAlike
% 3.0 Germany License (http://creativecommons.org/licenses/by-sa/3.0/de/): Feel
% free 'to share (to copy, distribute and transmit)' or 'to remix (to adapt)'
% it, if you '... distribute the resulting work under the same or similar
% license to this one' and if you respect how 'you must attribute the work in
% the manner specified by the author ...':
%
% In an internet based reuse please link the reused parts to www.fodina.de and
% mention the original author Karsten Reincke in a suitable manner. In a
% paper-like reuse please insert a short hint to www.fodina.de and to the
% original author, Karsten Reincke, into your preface. For normal quotations
% please use the scientific standard to cite.
%
% [ Derived from 'mykeds Classical Scholar Research Framework' 
%   mykeds-CSR-framework (c) K. Reincke CC BY 3.0  http://www.mykeds.net/ ]
%

\author{Karsten Reincke\footnote{This text is licensed under the Creative
Commons Attribution-ShareAlike 3.0 Germany License
(http://creativecommons.org/licenses/by-sa/3.0/de/): Feel free \glqq{}to share
(to copy, distribute and transmit)\grqq{} or \glqq{}to remix (to adapt)\grqq{}
it, if you \glqq{}[\ldots] distribute the resulting work under the same or
similar license to this one\grqq{} and if you respect how \glqq{}you must
attribute the work in the manner specified by the author [\ldots]\grqq{}):
\newline
In an internet based reuse please link the reused parts to www.fodina.de and
mention the original author Karsten Reincke in a suitable manner. In a
paper-like reuse please insert a short hint to www.fodina.de and to the original
author, Karsten Reincke, into your preface. For normal quotations please use the
scientific standard to cite.
\newline
{ \tiny \itshape [derived from mykeds-CSR (= Classical Scholar
Research framework) \copyright K. Reincke CC BY 3.0 http://www.mykeds.net/)] }
}}


%\thanks{den Autoren von KOMA-Script und denen von Jurabib}
\maketitle
%%-- end(titlepage)

% fodina humanities 'for being included' snippet template
%
% (c) Karsten Reincke, Frankfurt a.M. 2010, 2011, ff.
%
% This LaTeX-File is licensed under the Creative Commons Attribution-ShareAlike
% 3.0 Germany License (http://creativecommons.org/licenses/by-sa/3.0/de/): Feel
% free 'to share (to copy, distribute and transmit)' or 'to remix (to adapt)'
% it, if you '... distribute the resulting work under the same or similar
% license to this one' and if you respect how 'you must attribute the work in
% the manner specified by the author ...':
%
% In an internet based reuse please link the reused parts to www.fodina.de and
% mention the original author Karsten Reincke in a suitable manner. In a
% paper-like reuse please insert a short hint to www.fodina.de and to the
% original author, Karsten Reincke, into your preface. For normal quotations
% please use the scientific standard to cite.
%


\begin{abstract}
\noindent \itshape
Handling bibliographic references also influences the reception: if it's good it
makes the reading easier. One example for such an handling is the style of
classical scholar research and its' annotations presented in footnotes
(respectively endnotes). This little article demonstrates this classical scholar
style (respectively the style of the humanistics) -realized with \emph{LaTeX},
\emph{BibTeX}, \emph{komascript} and \emph{jurabib}.
\end{abstract}






%\footnotesize
%\tableofcontents
%\normalsize

% fodina humanities 'for being included' snippet template
%
% (c) Karsten Reincke, Frankfurt a.M. 2010, 2011, ff.
%
% This LaTeX-File is licensed under the Creative Commons Attribution-ShareAlike
% 3.0 Germany License (http://creativecommons.org/licenses/by-sa/3.0/de/): Feel
% free 'to share (to copy, distribute and transmit)' or 'to remix (to adapt)'
% it, if you '... distribute the resulting work under the same or similar
% license to this one' and if you respect how 'you must attribute the work in
% the manner specified by the author ...':
%
% In an internet based reuse please link the reused parts to www.fodina.de and
% mention the original author Karsten Reincke in a suitable manner. In a
% paper-like reuse please insert a short hint to www.fodina.de and to the
% original author, Karsten Reincke, into your preface. For normal quotations
% please use the scientific standard to cite.
%

\section{The Annotated Text}

A scientific paper\footnote{This English text is an extract of the longer
German version. The English text 'only' demonstrates the
classical sholar style (of humanities), The German text also explains why
this style is (a somewhat) better than the style recommended by the \textit{MLA
Handbook for Writers of Research Papers}. Both, the English and the German 
Version of \textit{Service For Readers and Scholars} are
generated on the base of the \textit{mycsrf-framework}. It's an Open
(Source) Document and can freely be downloaded under
http://github.com/kreincke/mycsrf} written in the \textit{Classical Scholar 
Research} style which not only wants to argue for a new position or insight
but to offer it's reader the possibility to adopt the research history by the
way. The history of humanities is the history of the secondary literature. Hence
the footnotes in the \textit{Classical Scholar Research} style present all
information about a work if it's quoted for the first time. If it is quoted
again, it's referred by the short title of the bib-file\footnote{I prefer the
pattern 'Author-Name: Short-Title, Year'. But I didn't find any solution to
convince jurabib to do this automatically. Therefore in each field 'shorttile'
of my bib-files I append at the real short title a comma followed by the year. If
anyone knows a better solution I would be glad to get a message from him.}. If
it is cited multiply - directly in a row of notes on the same page, then and
only then the shortcuts \textit{id.} and \textit{ibid.} or \textit{l.c.} should
be used. Let me demonstrate what this means:

\begin{itemize}
  \item A \textit{book}\footcite[cf.][123]{AllHen2008a} is quoted for the first time.
  \item A \textit{proceedings}\footcite[cf.][234]{Brachman1985a} is quoted for the first time.
  \item An \textit{inproceedings}\footcite[cf.][345]{Hays1985a}is quoted for the first time.
  \item An \textit{article of a journal}\footcite[cf.][456]{McCarthy1980a} is quoted for
  the first time.
  \item A \textit{book}\footcite[cf.][123]{AllHen2008a} is quoted for the second time.
  \item A \textit{proceedings}\footcite[cf.][234]{Brachman1985a} is quoted for the second
  time.
  \item An \textit{inproceedings}\footcite[cf.][345]{Hays1985a}is quoted for the
  second time.
  \item An \textit{article of a journal}\footcite[cf.][456]{McCarthy1980a} is quoted for the second time.
  \item A sophisticated book\footcite[cf.][567]{KantKdV1974} is quoted for the first time.
  \item Now - directly following - another page of this sophisticated
  book\footcite[cf.][678]{KantKdV1974} is quoted. 
  \item Now - again directly following - the same page of this sophisticated
  book\footcite[cf.][678]{KantKdV1974} is quoted again.
  \item Now another complex book of the same
  author\footcite[cf.][789]{KantKdU1974} is quoted for the first time.
  \item And now the first sophisticated book of the same
  author\footcite[cf.][789]{KantKdV1974} is quoted again.
  \item Sometimes it's better to note the bibliographic data of a book
  (collection, proceedings) [which covers / contains articles / parts of
  different authors] not as an autonomous bitex data set but as an inline part of
  the covered article. In this case the quoted article\footcite[cf.][23]{RotCum2011a} will be mentioned as integrated set of data at both places,
  on the page quoting the article and in the bibliography.
\end{itemize}






This is the end of the text, the end of the pure text. But now the annotations
and references follow each of them in separat section - hence in the annotations
are presented in endnotes.

\small
\input{btexmat/fodinaNclAbbreviationsInc}
\input{btexmat/fodinaNclAbbreviationsEnInc}
\input{btexmat/fodinaNclJournalsInc}
\printnomenclature

\theendnotes
\bibliography{bibfiles/fodinaHumanitiesExEn}
\end{document}
