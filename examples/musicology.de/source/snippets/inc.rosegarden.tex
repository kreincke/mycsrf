% mycsrf 'for beeing included' snippet template
%
% (c) Karsten Reincke, Frankfurt a.M. 2012, ff.
%
% This text is licensed under the Creative Commons Attribution 3.0 Germany
% License (http://creativecommons.org/licenses/by/3.0/de/): Feel free to share
% (to copy, distribute and transmit) or to remix (to adapt) it, if you respect
% how you must attribute the work in the manner specified by the author(s):
% \newline
% In an internet based reuse please link the reused parts to mycsrf.fodina.de
% and mention the original author Karsten Reincke in a suitable manner. In a
% paper-like reuse please insert a short hint to mycsrf.fodina.de and to the
% original author, Karsten Reincke, into your preface. For normal quotations
% please use the scientific standard to cite

%% use all entries of the bibliography

\subsection{Rosegarden ($\bigstar\bigstar\bigstar$)}

\parpic(1cm,1cm)[r][t]{\includegraphics[width=1cm]{logos/rosegarden-300dpi.png}}
\label{Rosegarden}\acc{Rosegarden} versteht sich als Kompositionsumgebung, die
um einen 'MIDI Sequencer' herum aufgebaut worden sei und dabei auch als
Notensatz- und digitales Audiosystem fungiere.\footcite[vgl.][\nopage
wp.]{Rosegarden2019a} Als \enquote{MIDI and audio sequencer} und \enquote{musical
notation editor} wolle es \enquote{das Tool der Wahl} für jene sein, die es
vorziehen, mittels Noten zu arbeiten:\footnote{Im Original: \enquote{to serve as
the sequencer of choice for users who prefer to work with music notation}
(\cite[vgl.][\nopage wp.]{Rosegarden2019c})}

\begin{quote}\enquote{\textit{Rosegarden allows you to record, arrange, and compose
music, in the shape of traditional score or MIDI data, or of audio files either
imported or recorded from a microphone, guitar or whatever audio source you care
to specify.}}\footcite[vgl.][\nopage wp.]{Rosegarden2019c} \end{quote}

Man könne mit \acc{Rosegarden} Musik schreiben, editieren oder komponieren,
diese synthetisieren, mit Effekten anreichern oder abmischen, um sie schließlich
auf CD zu brennen. Und nicht zuletzt biete \acc{Rosegarden} eben den
\enquote{[\ldots] well-rounded notation editing support for high quality printed
output via LilyPond}.\footcite[vgl.][\nopage wp.]{Rosegarden2019c}

Stellt man dem die These zur Seite, dass der \enquote{Kern eines Sequenzers
[\ldots] die Speicherung und Über\-mitt\-lung einer Partitur an einen
Tonerzeuger (sei)}, die \enquote{ [\ldots] in einem maschinenlesbaren Format
(vorliege)}, und dass diese Partitur \enquote{[\ldots] Tonhöhe, Tondauer und
ggf. weitere Aspekte der wiederzugebenden Noten einer oder mehrerer Stimmen in
ihrer zeitlichen Reihenfolge an ein Gerät (weitergäbe), das entsprechende Töne
(erzeuge)}\footcite[vgl.][\nopage wp.]{WpedSequencer2018a}, dann gewinnt man
damit eine recht genaue Vorstellung von dem, was \acc{Rosegarden} leisten will:
Es erlaubt den Import von \acc{MIDI}-Aufnahmen, bietet die Möglichkeit, diese zu
korrigieren, zu modifizieren, zu arrangieren und wieder
abzuspielen\footcite[vgl.][\nopage wp.]{Rosegarden2019c}. Ein Seiteneffekt des
Angebots ist, dass diese Modifikationen auch visuell über einen Noteneditor
erfolgen können.\footcite[vgl.][\nopage wp.]{Rosegarden2019d}

Als Open-Source-Software wird der Quelltext von \acc{Rosegarden} öffentlicht
gehostet und weiterentwickelt.\footnote{\cite[vgl.][\nopage
wp.]{Rosegarden2019e}. Die Projektseite gibt an, das Programm werde unter der
GPL-2.0 Lizenz distribuiert. Damit ist \acc{LilyPond} freie Software.} Die
Dokumentation wird als Wiki gepflegt\footcite[vgl.][\nopage wp.]{Rosegarden2019c}
und auch in thematischen Teilenbereichen bereitgestellt\footcite[vgl.][\nopage
wp.]{Rosegarden2019d}. Daneben gibt es noch spezielle
Tutorials\footcite[vgl.][\nopage wp.]{Rosegarden2019b}, die ausgehend von einem
bestimmten Aspekt in die Nutzung von \acc{Rosegarden}
einführen.\footcite[vgl.][\nopage wp.]{McIntyre2008a}

Vom Format her erlaubt es \acc{Rosegarden}, außer dem eigenen Dateityp auch
\acc{MIDI}- und \acc{MusicXML}-Dateien zu öffnen bzw. zu importieren. Beim Export
wird u.a. noch das \acc{LilyPond}-Format angeboten. Startet man einen
\acc{MIDI}-Server -- etwa \acc{timidity}\footnote{über eine Shell per
\texttt{timidity -iA}} --, bevor man \acc{Rosegarden} aufruft, können die
geladenen Noten außerdem erfolgreich abgespielt werden.

Unsere Referenzkadenz II vermag \acc{Rosegarden} problemlos als MusicXML-Datei
zu lesen, sofern diese keinen Text, also keine Harmonierungssymbole enhält.
Selektiert man beide Tracks und ruft über das Menue die Notenvisualierung
auf\footnote{\texttt{Select/Edit With/Open in Notation Editor}}, wird der
Notentext auf dem Bildschirm angezeigt:

\begin{center}
\includegraphics[width=0.9\textwidth]{frontends/rosegarden/rosegarden-cadenca2-300dpi.png}
\end{center}

Das Editieren des Notentextes ist gewöhnungsbedürftig, aber nicht unmöglich.
\acc{LilyPond}-Code kann nicht direkt eingegeben werden. Damit steht unsere
kleine Zusatzbibliothek für die Harmonieanalyse nicht zu Verfügung, allenfalls
die Umfunktionierung die normalen Liedtextfunktionalität. Wenn man unsere
Referenzkadenz als \acc{LilyPond}-Code exportiert, kann man das Ergebnis ohne
Abstriche z.B. in \acc{Frescobaldi} wieder einlesen.

Für die diejenigen, die ihre Musik eher über \acc{MIDI}-Kanäle eingeben wollen,
ist Rosegarden ein hervoragendes Frontend, ist doch seine optische Erscheinung
auf die spurenorientierte Bearbeitung von Musik auslegt. Die Funktion zum
Notensatz wirkt da etwas aufgesetzt. Für einen eher notenorientierten
Musikwissenschaftler wird Rosegarden nicht wirklich das Mittel der Wahl darstellen,
auch wenn es das, was es leisten will, ausgezeichnet tut. Insofern geben wir 
Rosengarden -- nur aus unserem Kontext heraus -- drei Sterne.


% this is only inserted to eject fault messages in texlipse
%\bibliography{../bib/literature}
