

% (c) Karsten Reincke, Frankfurt a.M. 2012, ff.
%
% This text is licensed under the Creative Commons Attribution 3.0 Germany
% License (http://creativecommons.org/licenses/by/3.0/de/): Feel free to share
% (to copy, distribute and transmit) or to remix (to adapt) it, if you respect
% how you must attribute the work in the manner specified by the author(s):
% \newline
% In an internet based reuse please link the reused parts to mycsrf.fodina.de
% and mention the original author Karsten Reincke in a suitable manner. In a
% paper-like reuse please insert a short hint to mycsrf.fodina.de and to the
% original author, Karsten Reincke, into your preface. For normal quotations
% please use the scientific standard to cite
%

Ich selbst hatte mir zu Beginn eines größeren musikwissenschaftlichen Projektes gewünscht, ein solches Tutorial zu haben: \acc{\LaTeX}, \acc{Bib\TeX}\ und \acc{JabRef} waren mir schon vertraut. Meinen optimalen Zitierstil für das Schreiben geisteswissenschaftlicher Arbeiten hatte ich bereits konfiguriert\footcite[vgl.][\nopage wp.]{Reincke2018a} und dokumentiert\footcite[vgl][2ff]{Reincke2018b}. Unklar war mir 'nur', wie man Notentexte und musikalische Analysen in \LaTeX-Texte einbindet.

Das sollte eigentlich nicht kompliziert sein. Man bräuchte dazu doch nur ein No\-ta\-tions\-system, das Notentext erfasst und das -- als Teil des \LaTeX-Quelltextes -- das Notenbild in das eigentlich Werk hinein generiert. Leider schwiegen sich meine sehr guten, einschlägigen \LaTeX-Bücher dazu aus\footcite[vgl.][vi ff, insbesondere 905 u. 909: das umfangreiche Register erwähnt weder Musik im allgemeinen noch LilyPond oder MusiX\TeX\ im Besonderen]{Voss2012a}, selbst wenn sie auch Randbereiche behandelten\footcite[vgl.][vii ff, insbesondere 1080 u. 1087: auch dieses umfangreiche Register erwähnt weder Musik im allgemeinen noch LilyPond oder MusiX\TeX\ im Besonderen.]{MitGoo2005a}. Die entsprechende Internetrecherche überrollte mich dagegen: so viele Notationssysteme und Tools, aber kein systematischer Überblick.\footnote{Rühmlich die Ausnahme von \cite[][\nopage wp.]{Thoma2018a}. Allerdings ging sie nicht in die Tiefe, die ich benötigte.}

Wollte ich meine Arbeit also nicht gefährden und Sackgassen vermeiden, musste ich die gegebenen Möglichkeiten zuerst sichten. Andernfalls wäre ich Gefahr gelaufen, zuletzt doch 'aufs falsche Pferd gesetzt' zu haben. Was ich brauchte war eine technisch fundierte 'Landkarte' der Methoden und Tools, die mir den besten Weg weisen konnte. Und so traf mich -- wieder einmal -- die Erkenntnis:

\begin{quote}\textit{Was man im Internet nicht findet, muss man selbst hineinstellen - am Besten unter einer Lizenz, die jedem die freie Wieder- und Weiterverwendbarkeit garantiert.} \end{quote}

 % this is only inserted to eject fault messages in texlipse
%\bibliography{../bib/literature}
