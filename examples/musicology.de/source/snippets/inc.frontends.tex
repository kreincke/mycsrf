% mycsrf 'for beeing included' snippet template
%
% (c) Karsten Reincke, Frankfurt a.M. 2012, ff.
%
% This text is licensed under the Creative Commons Attribution 3.0 Germany
% License (http://creativecommons.org/licenses/by/3.0/de/): Feel free to share
% (to copy, distribute and transmit) or to remix (to adapt) it, if you respect
% how you must attribute the work in the manner specified by the author(s):
% \newline
% In an internet based reuse please link the reused parts to mycsrf.fodina.de
% and mention the original author Karsten Reincke in a suitable manner. In a
% paper-like reuse please insert a short hint to mycsrf.fodina.de and to the
% original author, Karsten Reincke, into your preface. For normal quotations
% please use the scientific standard to cite
%


%% use all entries of the bibliography

\subsection{Frontends}

Eine umfangreiche Sichtung\footcite[vgl.][\nopage wp]{WpedNotensatz2019a} von
Notensatzprogrammen listet die freien Frontends \textit{Aria Maestosa},
\textit{Brahms}, \textit{Canorus}, \textit{Denemo}, \textit{Laborejo},
\textit{Mup}, \textit{MuseScore}, \textit{NoteEdit}, \textit{NtEd},
\textit{Rosegarden} auf. Außerdem findet man im Netz noch die Tools
\textit{Frescobaldi}, \textit{muX2d} und \textit{Elysium/Eclipse}.
Für die \acc{ABC-Notationsmethode} stehen auch einige Opensource-Frontends zur
Verfügung, etwa \acc{EasyABC} oder \acc{ABCJ}\footcite[vgl.][\nopage
wp]{Abc2018b}. Die wohl umfangreichste Sammlung von Musiksoftware liefert die
Site \acc{MusicXML}\footcite[vgl.][\nopage wp]{MusicCML2018b}, einfach weil sie
schlicht auflistet, welche Software MusicXML-Dateien liest, schreibt oder liest
und schreibt. Diese Liste enthält mithin auch andere Programme als solche für
den Notensatz. Und sie listet auch proprietäre Applikationen. Von den freien
Notensatzprogrammen dieser Liste evaluieren wir noch \textit{Free Clef},
\textit{MusEdit}, \textit{Audimus Notes}, \textit{Ptolemaic},
 und \textit{Audoria}\footnote{Die Evaluation von Programmen, die dezidiert für
 Android oder iOS oder Windows gedacht sind und die spezielle Zwecke verfolgen,
 also etwa \acc{Finale Notepad}, \acc{Ossia Viewer}, \acc{Candezii},
 \acc{Crescendo}, acc{Jniz} \acc{Notation Pad} \acc{Score Creator} etc.
 überlassen wir mit einer gewissen Willkür einer späteren Version unseres Textes.}.


\textit{Aria Maestosa}, \textit{Brahms} und \textit{Rosegarden}
sind in erster Linie \textsf{Sequenzer}, \textit{Frescobaldi} und
\textit{Elysium/Eclipse} bieten 'nur' eine textuelle Eingabeschnittstelle,
machen das Ergebnis der Eingabe aber direkt als Bild sichtbar. Die anderen
dürfen als visuelle Notensatzprogramme angesehen werden.

Systematisch kann man die Lage so darstellen:
 
\begin{center}\scriptsize
\begin{tabulary}{15cm}{R||C|C|C|C|C|C||C|C||C|C|C|C|C|C|C|C|C|}
\hline
Frontend & 
  \multicolumn{6}{c||}{Import} & 
  \multicolumn{2}{c||}{Change} & 
  \multicolumn{9}{c|}{Export} \\
\hline
Programm & 
  \rotatebox{90}{ABC} & 
  \rotatebox{90}{LilyPond} & 
  \rotatebox{90}{Midi} & 
  \rotatebox{90}{MusicXML} & 
  \rotatebox{90}{MusiX\TeX} & 
  \rotatebox{90}{PMX} &
  \rotatebox{90}{graphisch} & \rotatebox{90}{textuell} &
  \rotatebox{90}{ABC} & 
  \rotatebox{90}{LilyPond} & 
  \rotatebox{90}{Midi} & 
  \rotatebox{90}{MusicXML} & 
  \rotatebox{90}{MusiX\TeX} & 
  \rotatebox{90}{PMX} &  
  \rotatebox{90}{PDF} &  
  \rotatebox{90}{PS} &  
  \rotatebox{90}{PNG $\lor$ JPEG $\lor$ SVG}
\\
\hline
\hline
ABCJ & 
  1 & 2 & 3 & 4 & 5 & 6 &
  7 & 8 & 
  9 & A & B & C & D & E & F & G & H \\
\hline
ABC4J & 
  1 & 2 & 3 & 4 & 5 & 6 &
  7 & 8 & 
  9 & A & B & C & D & E & F & G & H \\
\hline
Andoria & 
  1 & 2 & 3 & 4 & 5 & 6 &
  7 & 8 & 
  9 & A & B & C & D & E & F & G & H \\
\hline
Aria Mae. & 
  1 & 2 & 3 & 4 & 5 & 6 &
  7 & 8 & 
  9 & A & B & C & D & E & F & G & H \\
\hline
Audimus N. & 
  1 & 2 & 3 & 4 & 5 & 6 &
  7 & 8 & 
  9 & A & B & C & D & E & F & G & H \\
\hline
Brahms & 
  1 & 2 & 3 & 4 & 5 & 6 &
  7 & 8 & 
  9 & A & B & C & D & E & F & G & H \\
\hline
Canorus & 
 1 & 2 & \checkmark & \checkmark & 5 & 6 & 
 \checkmark & 8 & 
 9 & \checkmark & \checkmark & \checkmark & D & E & \checkmark & G & \checkmark \\
\hline
Denemo & 
  1 & 2 & 3 & 4 & 5 & 6 &
  7 & 8 & 
  9 & A & B & C & D & E & F & G & H \\
\hline
EasyABC & 
  1 & 2 & 3 & 4 & 5 & 6 &
  7 & 8 & 
  9 & A & B & C & D & E & F & G & H \\
\hline
Elysium & 
  1 & 2 & 3 & 4 & 5 & 6 &
  7 & 8 & 
  9 & A & B & C & D & E & F & G & H \\
\hline
Free Clef & 
  1 & 2 & 3 & 4 & 5 & 6 &
  7 & 8 & 
  9 & A & B & C & D & E & F & G & H \\
\hline
Frescobaldi & 
  1 & 2 & 3 & 4 & 5 & 6 &
  7 & 8 & 
  9 & A & B & C & D & E & F & G & H \\
\hline
Laborejo & 
  1 & 2 & 3 & 4 & 5 & 6 &
  7 & 8 & 
  9 & A & B & C & D & E & F & G & H \\
\hline
MusEdit & 
  1 & 2 & 3 & 4 & 5 & 6 &
  7 & 8 & 
  9 & A & B & C & D & E & F & G & H \\
\hline
MuseScore & 
  1 & 2 & 3 & 4 & 5 & 6 &
  7 & 8 & 
  9 & A & B & C & D & E & F & G & H \\
\hline
muX2D & 
  1 & 2 & 3 & 4 & 5 & 6 &
  7 & 8 & 
  9 & A & B & C & D & E & F & G & H \\
\hline
NoteEdit & $\neg$ &$\neg$ &$\neg$ &$\neg$ &$\neg$ &$\neg$ & 
$\neg$& $\neg$& 
$\neg$& $\neg$& $\neg$& $\neg$& $\neg$& $\neg$&  $\neg$& $\neg$ & $\neg$ \\
\hline 
NtEd & $\neg$ & $\neg$ & \checkmark & \checkmark & $\neg$ & $\neg$ &
 \checkmark & $\neg$ &
  $\neg$ & \checkmark & \checkmark & $\neg$ & $\neg$ & $\neg$ & \checkmark & \checkmark  & \checkmark \\
\hline
Ptolemaic & 
  1 & 2 & 3 & 4 & 5 & 6 &
  7 & 8 & 
  9 & A & B & C & D & E & F & G & H \\
\hline
\end{tabulary}
\end{center}
  
Eine solche Fülle von Möglichkeiten fordert geradezu dazu auf, praktisch zu
überprüfen, welche den Anforderungen von Musikwissenschaftlern (am besten)
gerecht werden. Deshalb werden wir testen, wie diese Tools mit der Referenzkadenz II
umgehen können, in wie weit sie die Symbole der Harmonieanalyse aufzunehmen
vermögen und ob sie diesen Inhalt in einem Format abspeichern
können\footnote{Konkret werden wir unsere Referenzkadenz-II jeweils mit den
\acc{LilyPond}-kompatiblen Frontends erfassen, das Result von dort aus als
\acc{LilyPond}-Datei exportieren und diese wiederum vom Eclipse-Plugin
\acc{Elysium} einlesen lassen. \acc{Elysium} leitet aus dieser ursprünglich
exportierten Datei automatisch eine PDF-Datei ab, die genauso aussehen sollte,
wie das Original. Stimmt also diese Datei mit den Erwartungen überein,
betrachten wir den ursprünglichen Export als verifziert.\label{ExportVerifikation}
Die anderen Exportformate werden wir entsprechend überprüfen. Einzelheiten dazu
vermerken wir bei den entsprechenden Fällen.}.

% this is only inserted to eject fault messages in texlipse
%\bibliography{../bib/literature}
