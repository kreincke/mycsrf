% mycsrf 'for beeing included' snippet template
%
% (c) Karsten Reincke, Frankfurt a.M. 2012, ff.
%
% This text is licensed under the Creative Commons Attribution 3.0 Germany
% License (http://creativecommons.org/licenses/by/3.0/de/): Feel free to share
% (to copy, distribute and transmit) or to remix (to adapt) it, if you respect
% how you must attribute the work in the manner specified by the author(s):
% \newline
% In an internet based reuse please link the reused parts to mycsrf.fodina.de
% and mention the original author Karsten Reincke in a suitable manner. In a
% paper-like reuse please insert a short hint to mycsrf.fodina.de and to the
% original author, Karsten Reincke, into your preface. For normal quotations
% please use the scientific standard to cite
%


%% use all entries of the bibliography

\subsection{Lilypond}

\emph{LilyPond} möchte guten \enquote{Notensatz für jedermann} anbieten: Als
elektronisches \enquote{Notensatzsystem} wolle es \enquote{[\ldots] Notendruck
in (bester) Qualität} ermöglichen, also \enquote{[\ldots] die Ästhetik
handgestochenen traditionellen Notensatzes mit computergesetzten Noten [\ldots]
erreichen}\footcite[vgl.][\nopage wp]{LilyPond2018a}. In einem besonderen
Artikel haben die LilyPond-Entwickler dargestellt, was das systemisch
bedeutet\footcite[vgl.][5ff]{LilyPond2018d} und welchen Konsequenzen sich daraus
für ein Notensatzprogramm ergeben\footcite[vgl.][8ff]{LilyPond2018d}. Der daraus
erwachsende Anspruch ist hoch:

\begin{quote}\begin{em}
  \enquote{LilyPond wurde geschaffen, um die Probleme zu lösen, die wir in
  existierenden Programmen gefunden haben und um schöne Noten zu schaffen, die
  die besten handgestochenen Partituren imitieren.}\footcite[vgl.][2]{LilyPond2018d}
\end{em}\end{quote}

Wer die entsprechenden Techniken erfolgreich anwenden will, kann auf ein einfach
strukturiertes Lerntutorial\footcite[vgl.][20ff]{LilyPond2018b} und ein kürzeres
Nutzungshandbuch\footcite[vgl.][1ff]{LilyPond2018e} zurückgreifen. Letztlich
wird er sich auch das umfangreiche
Referenzhandbuch\footcite[vgl.][1ff]{LilyPond2018c} bereitlegen wollen.

Wie die bisher diskutierten Systeme erwartet auch \emph{LilyPond}, dass man Code
schreibt, keine Noten: Hier wie da ist der Texteditor das bevorzugte Werkzeug,
um Musik im entsprechenden 'Dialekt' zu notieren. Trotzdem gibt es Unterschiede,
die über die bloße Syntax hinausgehen:

Die wichtigste Eigenart dürfte sein, dass Lilypond konsequent zwischen Musik und
Druck unterscheidet: Wer in D-Dur ein \emph{fis} einfügen möchte, kann sich hier
nicht auf die zu Beginn spezifizierte Tonart 'berufen', er muss trotzdem
\texttt{fis} tippen, nicht \texttt{f}, und zwar an jeder Stelle, wo er
\emph{fis} meint. Diese Abkehr von der Tradition hat einen gewichtigen Vorteil:
Lilypond kann bei alterierten Passagen die nötigen Vorzeichen automatisch
setzen. In \emph{g-moll} erhält das \emph{fis} ein Kreuz, in D-Dur
nicht\footcite[vgl.][21]{LilyPond2018b}.

Die augenfälligste Besonderheit dürfte jedoch sein, dass \emph{LilyPond} seine
Elemente konsequent in einer 1:n-Beziehung anordnet: Das Notenheft besteht aus
einem oder mehreren Stücken, das Stück besteht aus einem oder mehreren
Notensystemen, ein Notensystem besteht aus einer oder mehrerer Stimmen, die
Stimme kann solistisch oder akkordisch sein. Das Datenmodell ist mithin als Baum
ausgelegt. Und syntaktisch haben die Ebenen je eigene Markanten. Das macht das
Lesen und Verstehen von \emph{LilyPond}-Code auf Dauer einfacher, es entsteht ein
klarerer Sourcetext\footcite[vgl.][40ff]{LilyPond2018b}.

Systemisch gesehen hat LilyPond (heute) nichts (mehr) \LaTeX\ , MusiX\TeX\ oder
\TeX\ zu tun: es nutzt seine eigene Eingabesprache und seine eigene Maschine zum
Erzeugen des Notenbildes: Als \enquote{Standardausgabeformat} -- heißt es --
seien \emph{PDF}\footnote{Portable Document Format} und
\emph{PS}\footnote{Postscript} gesetzt; außerdem könnten
\emph{SVG\footnote{Scalable Vector Graphics}-}, \emph{EPS\footnote{Encapsulated
PostScript}-} und \emph{PNG\footnote{Portable Network Graphics}-}Dateien erzeugt
werden\footcite[vgl.][481]{LilyPond2018c}.

\subsubsection{Technische Voraussetzungen}

\emph{LilyPond} sagt selbst, dass man Notenbeispiele in Form von Graphiken auch
manuell in den \LaTeX-Text einfügen könne, einfach in dem man -- zuerst und
unabhängig von \LaTeX -- die Graphiken erzeuge und sie danach mit \LaTeX-Mitteln
einbinde\footcite[vgl.][20]{LilyPond2018e}. Bei vielen Notenbeispielen kann das
allerdings aufwendig werden, insbesondere wenn man manuell die Länge der
Notenzeilen und die Graphikbreite auf die gewünschte Zeilenlänge des Dokumentes
ausrichten muss. 

Deshalb bietet \emph{LilyPond} -- neben \texttt{lilypond} als Tool zur Erzeugung
ganzer Notenblätter in den geannten Formaten\footnote{samt aller anderen
Outputformate wie \emph{midi} u.Ä.m.} -- auch das Tool \texttt{lilypond-book}
an: es \enquote{automatisiert} die manuelle Integration, in dem es die
\enquote{[\ldots] Musik-Schnipsel aus Ihrem Dokument (extrahiert), [\ldots]
\texttt{lilypond} (aufruft) und [\ldots] die resultierenden Bilder in Ihr
Dokument (einfügt)}, wobei es \enquote{[\ldots] die Länge der Zeilen und die
Schriftgröße dabei [automatisch] (dem) Dokument
(anpasst}\footcite[vgl.][20]{LilyPond2018e}.

Damit wissen wir, dass man auch hier einiges vorzubereiten hat, wenn man
\emph{LilyPond} erfolgreich verwenden will:

$\RHD$ Zunächst muss -- ganz unabhängig von \LaTeX\ -- \emph{LilyPond}
installiert werden\footnote{Unter Ubuntu: \texttt{sudo apt-get lilypond
lilypond-data}}. Dieses Paket stellt dann auch LilyPond-Book bereit.
  
$\RHD$ Im Gegensatz zu \emph{ABC} oder \emph{MusiX\TeX} muss man bei
\emph{LilyPond} kein Paket in die \LaTeX-Präambel einbinden. Denn
\texttt{lilypond-book} muss immer zuerst und unabhängig von \texttt{latex} bzw.
\texttt{pdflatex} aufgerufen werden: Es generiert erst den eigentlichen
\LaTeX-Code, der dann keine \verb|\begin{lilypond}...\end{lilypond}|-Umgebungen
mehr enthält. Also kann -- sozusagen direkt nach der Installation -- der
\emph{LilyPond}-Quelltext eines jeden Notenbeispiels in je einer eigenen
(virtuellen) Umgebung \verb|\begin{lilypond}...\end{lilypond}| zusammengestellt
werden. Virtuell sind diese Umgebungen insofern, als \LaTeX ja nichts von
\emph{LilyPond} weiß.

$\RHD$ Schließlich muss man noch organisieren, dass den eigentlichen
\LaTeX-Aufrufen ein \emph{lilypond-book}-Aufruf vorausgeht. Das kann wieder in
einem Makefile organisiert werden.

Leider steckt der Teufel dabei -- wie so oft -- im Detail: 

\emph{lilypond-book} nimmt eine -- wie wir jetzt wissen -- gewissermaßen
'unechte' \LaTeX-Datei mit \emph{lilypond}-Sektionen, erzeugt die entsprechenden
Graphiken und ersetzt die \emph{lilypond}-Sektionen durch die passenden
'include-Graphik'-Befehle. Das ist der Grund, warum \emph{lilypond-book} immer
als erstes aufgerufen werden muss.

Ruft man \emph{lilypond-book} ohne weitere Parameter für eine 'unechte
\LaTeX-Datei' mit der Extension \emph{.tex} auf, beschwert es sich jedoch, dass
es seine Inputdatei überschreiben müsste und verweigert die Weiterarbeit. Dies
Problem wird gelöst, indem man es mit der Option --out und einem Namen aufruft.
Dann erzeugt \emph{lilypond-book} einen Ordner dieses Namens und sammelt darin
alle Materialien ein, die es für eine \LaTeX-compilierbare Version benötigt.

Unglücklicherweise findet \emph{lilypond-book} nicht alles, was tatsächlich
benötigt wird: \emph{lilypond-book} evaluiert und bearbeitet zwar sehr
erfolgreich alle Dateien mit der Extension \emph{.tex}, und zwar insbesondere
auch die, die per \emph{input}-Befehl in die Hauptdatei eingebunden sind. Und
diese Dateien kopiert es auch --- ggfls. überarbeitet -- in den Zielordner. Es
gibt alledings Dateien, die der \LaTeX-Durchgang benötigt, die aber von
\emph{lilypond-book} nicht entdeckt und also auch nicht in den Zielordner
kopiert werden. Prominentestes Beispiel sind die bib-Files.

Deshalb muss der Nutzer von \emph{lilypond-book} -- sozusagen manuell -- die
fehlenden Dateien in den von \emph{lilypond-book} erzeugten Arbeitsordner kopieren,
bevor er den 'normalen' \LaTeX-Erzegungsprozess aufruft.

Eine entsprechendes Makefile könnte so aussehen:

\begin{verbatim}
tbd
\end{verbatim}


\subsubsection{Kadenz I}

Und damit können wir die Früchte der Arbeit ernten. Das mittlerweile Vertraute
Beispiel der Grabner-Kadenz sieht -- mit \emph{LilyPond} erzeugt -- so aus:

\begin{center}
\begin{lilypond}
\version "2.18.2"
\header { tagline = "" }
\score {
  \new Staff {
    \relative c' { 
      \time 3/1
      <c  e g>1 _\markup {I} ^\markup {T}
      <f a c>  _\markup {IV} ^\markup {S}
      <g b d>  _\markup {V} ^\markup {D}
      |
      <a, c e> _"I"         ^"T"
      <d f a>  _"IV"        ^"S"
      <e gis b> _"V"        ^"D"
      \bar "||"
    }   
  }
  \layout {
    \context {
      \Staff
        \remove Time_signature_engraver
    }
  }
}
\end{lilypond}
\cad{I}{LilyPond}
\end{center}

Und der entsprechende Quellcode sieht so aus:
\begin{verbatim}
\begin{lilypond}
\version "2.18.2"
\header { tagline = "" }
\score {
  \new Staff {
    \relative c' { 
      \time 3/1
      <c  e g>1 _\markup {I} ^\markup {T}
      <f a c>  _\markup {IV} ^\markup {S}
      <g b d>  _\markup {V} ^\markup {D}
      |
      <a, c e> _"I"         ^"T"
      <d f a>  _"IV"        ^"S"
      <e gis b> _"V"        ^"D"
      \bar "||"
    }   
  }
  \layout {
    \context {
      \Staff
        \remove Time_signature_engraver
    }
  }
}
\end{lilypond}
\end{verbatim}


\subsubsection{Kadenz II}

Die zweite Kadenz sieht dann so aus

\begin{center}
\begin{lilypond}
\version "2.18.2"
\header { tagline = "" }
\score {
  \new StaffGroup {
    \time 4/2
    <<
      \new Staff {
        \relative d' {
          \clef "treble"
          \key d \major  
          \stemUp
          < fis  d'>2 
          < fis  dis'>2 
          < b  e>2 
          < b  e>2
          |
          < b fis'>2 
          < e gis >2 
          < e a >2
          < a fis>2       
          \bar "||"
        }   
      }
      \new Staff {
        \relative d { 
          \clef "bass"
          \key d \major  
          \stemDown
          < d a'>2 ^\markup {T}
          < b a'>2 ^\markup {(D7)}
          < d g>2 ^\markup {Sp7}
          < cis g'>2 ^\markup {D79}
          |
          < d fis>2 ^\markup {Tp}
          < b d>2  ^\markup {DD7}
          <<
            { a2 ^\markup {D4>3}}
            { d4( cis4) }
          >> 
          < d, d'>2 ^\markup {T}      
          \bar "||"
        }   
      }
    >>
  }
}
\end{lilypond}
\cad{II}{LilyPond}
\end{center}
Und der entsprechende Quellcode sieht so aus:
\begin{verbatim}
\begin{lilypond}
\version "2.18.2"
\header { tagline = "" }
\score {
  \new StaffGroup {
    \time 4/2
    <<
      \new Staff {
        \relative d' {
          \clef "treble"
          \key d \major  
          \stemUp
          < fis  d'>2 
          < fis  dis'>2 
          < b  e>2 
          < b  e>2
          |
          < b fis'>2 
          < e gis >2 
          < e a >2
          < a fis>2       
          \bar "||"
        }   
      }
      \new Staff {
        \relative d { 
          \clef "bass"
          \key d \major  
          \stemDown
          < d a'>2 ^\markup {T}
          < b a'>2 ^\markup {(D7)}
          < d g>2 ^\markup {Sp7}
          < cis g'>2 ^\markup {D79}
          |
          < d fis>2 ^\markup {Tp}
          < b d>2  ^\markup {DD7}
          <<
            { a2 ^\markup {D4>3}}
            { d4( cis4) }
          >> 
          < d, d'>2 ^\markup {T}      
          \bar "||"
        }   
      }
    >>
  }
}
\end{lilypond}
\end{verbatim}


\subsubsection{Kadenz III}

Und schließlich fehlt noch die dritte Kdenz in der Version, die \emph{LilyPond}
erzeugt:

\begin{lilypond}
\version "2.18.2"
\header { tagline = "" }
\score {
  \new StaffGroup {
    \time 5/8
    <<
      \new Staff {
        \relative c'' {
          \clef "treble"
          \key bes \major  
          \stemUp
          < bes d  f >4 < c  es f >4 < bes es bes'>8 |
          < c   es a >4 < a  f' a>4 < c   es f   >8 |         
          r8 < f, d' f >8 < a  c  es >8 < f bes d>4 |
          r8 < es a  c >4 < es a  c  >4|
          r8 < es a  c >8 < c' es f  >8 < bes d f >4 |
          r8  < g bes c >8 < es a c >8 <d f bes>4
        }   
      }
      \new Staff {
        \relative c { 
          \clef "bass"
          \key bes \major  
          \stemDown
            bes8[ bes'] a[ as] g16.[ ges32] |
            f8[ es] d8.[ des16] c16.[ a32] | 
            bes4  f'8 bes4 |
            r8 f8 c f,4  | 
            r8 f' a,8 bes4 | 
            r8  es8 f  bes4
          \bar "||"
        }   
      }
    >>
  }
}
\end{lilypond}
\cad{III}{LilyPond}

Für diese ist folgender Quellcode zuständig:

\begin{verbatim}
\begin{lilypond}
\version "2.18.2"
\header { tagline = "" }
\score {
  \new StaffGroup {
    \time 5/8
    <<
      \new Staff {
        \relative c'' {
          \clef "treble"
          \key bes \major  
          \stemUp
          < bes d  f >4 < c  es f >4 < bes es bes'>8 |
          < c   es a >4 < a  f' a>4 < c   es f   >8 |         
          r8 < f, d' f >8 < a  c  es >8 < f bes d>4 |
          r8 < es a  c >4 < es a  c  >4|
          r8 < es a  c >8 < c' es f  >8 < bes d f >4 |
          r8  < g bes c >8 < es a c >8 <d f bes>4
        }   
      }
      \new Staff {
        \relative c { 
          \clef "bass"
          \key bes \major  
          \stemDown
            bes8[ bes'] a[ as] g16.[ ges32] |
            f8[ es] d8.[ des16] c16.[ a32] | 
            bes4  f'8 bes4 |
            r8 f8 c f,4  | 
            r8 f' a,8 bes4 | 
            r8  es8 f  bes4
          \bar "||"
        }   
      }
    >>
  }
}
\end{lilypond}
\end{verbatim}

\subsubsection{Bewertung}


% \footcite[vgl.][]{LilyPond2018a}.



% this is only inserted to eject fault messages in texlipse
%\bibliography{../bib/literature}
