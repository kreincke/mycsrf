% mycsrf 'for beeing included' snippet template
%
% (c) Karsten Reincke, Frankfurt a.M. 2012, ff.
%
% This text is licensed under the Creative Commons Attribution 3.0 Germany
% License (http://creativecommons.org/licenses/by/3.0/de/): Feel free to share
% (to copy, distribute and transmit) or to remix (to adapt) it, if you respect
% how you must attribute the work in the manner specified by the author(s):
% \newline
% In an internet based reuse please link the reused parts to mycsrf.fodina.de
% and mention the original author Karsten Reincke in a suitable manner. In a
% paper-like reuse please insert a short hint to mycsrf.fodina.de and to the
% original author, Karsten Reincke, into your preface. For normal quotations
% please use the scientific standard to cite
%


\section{Exkurs: MIDI unter GNU/Linux resp. Ubuntu} \label{MIDI}

Dieser Übersicht zufolge können einige der Frontends für Notensatzsysteme mit \acc{MIDI}-Dateien umgehen. Das impliziert oft die Möglichkeit, sich auch die Musik hinter den Noten 'über das Frontend' anzuhören. Manche Systeme bringen dazu bereits alles Nötige mit, andere erwarten eine darauf ausgelegte Umgebung. Und bei letzteren haben manche Distributoren die angeforderten Tools sauber mit paketiert, andere lassen Lücken.

Wie man Noten hörbar macht, ist eigentlich nicht unser Thema. Allerdings spielt es -- sofern man nach dem besten Frontend für einen Musikwissenschaftler sucht -- durchaus eine Rolle, ob die Frontends der Notensatzsysteme wenigstens prinzipiell auch das Hören von Musik erlauben. Wir werden deshalb -- sehr kurz und kursorisch -- am Beispiel von Ubuntu erläutern\footnote{getestet für Ubuntu 18.04 und 20.04}, was notwendig ist, um \acc{MIDI}-Datei mit freier Software über das System erklingen zu lassen. Dabei zielen wir weder auf Vollständigkeit noch Adäquanz im Detail. Das Prinzip allerdings sollte so weit klar werden, dass jeder mit der Hilfe von 'Dr. Google' sein eigenes System anpassen kann.

\acc{MIDI} steht für \enquote{Musical Instrument Digital Interface}.
\acc{MIDI}-Dateien sind mithin keine Audiodateien im 'normalen' Sinne. Sie
bieten keine digitalisierten Akkustikdaten. Vielmehr enthalten sie
klangunabhängige Beschreibungen von 'Stimmen', die -- sozusagen 'willkürlich' --
einem Sound, einem Instrument zugeordnet werden. Sie sind \enquote{musikalische
Steuerinformationen für elektronische Instrumente}.\footcite[vgl. dazu][\nopage
wp.]{WpedMidi2019a} Um \acc{MIDI}-Datei abzuspielen, braucht man also einen
Player, der -- im Verbund mit dem Computersoundsystem -- nicht nur
digitalisierte 'Analogdaten' in reale Analogdaten umwandelt\footnote{Die
Widersprüchlichkeit dieses 'Wordings' lässt sich leicht auflösen:
Musik ist zuletzt ein Strom sich überlagernder Frequenzen. Sie zu digitalisieren
bedeutet, diesen Strom -- wie die Gurke für den Gurkensalat -- in Scheiben zu
zerschneiden: Jede Scheibe wird ein Tupel von Zahlen, der die je zum
Schnittzeitpunkt klingenden Frequenzen repräsentiert. Und wie beim Gurkensalat,
so auch bei der Musik: je feiner die Scheiben, desto besser das sensorische
Ergebnis, desto besser die \acc{digitalisierten Analogdaten}. Unabhängig davon
kann Musik aber auch auf einer Metaebene als System digital so beschrieben
werden, dass man daraus -- wie bei \acc{MIDI} -- unter Rückgriff auf externe
Zutaten wieder ein Strom \acc{digitalisierten Analogdaten} ableiten kann, die
dann ihrerseits über das Computersoundsystem zu 'realen' Analog'daten' werden.},
sondern der auch die soundunabhängige Beschreibung und eine Soundbibliothek
zusammenbringt, um daraus die finalen analogen Musikdaten zu erzeugen.

Unter GNU/Linux bringt das Soundsystem -- unabhängig davon, ob es rein auf
\acc{ALSA} basiert\footnote{$\rightarrow$
\href{https://de.wikipedia.org/wiki/Advanced{\_}Linux{\_}Sound{\_}Architecture}
{https://de.wikipedia.org/wiki/Advanced{\_}Linux{\_}Sound{\_}Architecture}} oder
dieses im Verbund mit \acc{Pulse-Audio}\footnote{$\rightarrow$
\href{https://www.freedesktop.org/wiki/Software/PulseAudio/}
{https://www.freedesktop.org/wiki/Software/PulseAudio/}} als
\enquote{Sound-Middleware}\footnote{$\rightarrow$
\href{https://de.wikipedia.org/wiki/PulseAudio}
{https://de.wikipedia.org/wiki/PulseAudio}} bereitstellt\footnote{$\rightarrow$
\href{https://wiki.ubuntuusers.de/Soundsystem/}{https://wiki.ubuntuusers.de/Soundsystem/}}
-- nicht von sich aus die Fähigkeit mit, \acc{MIDI}-Dateien abzuspielen.
Um das zu ermöglichen, kann man seinem Player --- wenn er es denn erlaubt -- ein
zusätzliches \acc{MIDI}-Plugin spendieren. Oder man installiert einen generell
nutzbaren \acc{MIDI}-Server. Für beides ein Beispiel:

Der \acc{vlc} ist ein sehr vielseitiger Mediaplayer.\footnote{$\rightarrow$
\href{https://www.videolan.org/vlc/}{https://www.videolan.org/vlc/}} Um ihn samt
\acc{MIDI}-Player-Plugin zu installieren, genügt ein Kommando. Außerdem sollte
man das Soundsystem noch um eine bessere Soundbibliothek erweitern:

\begin{verbatim}
sudo apt-get install vlc vlc-plugin-fluidsynth
sudo apt install fluid-soundfont-gm fluid-soundfont-gs
\end{verbatim}

Das reicht aber noch nicht, um auch in Zukunft alle \acc{MIDI}-Dateien automatisch vom \acc{vlc} abgespielt zu bekommen. Oft ist schon ein Player für alle Medien-Dateien zuständig und \acc{MIDI}-Dateien werden den Medien-Dateien zugerechnet, obwohl das Soundsystem -- wie wir gelernt haben -- von sich aus ja gar keine \acc{MIDI}-Dateien versteht. Also ist es reiner Zufall, ob die voreingestellte Defaultapplikation \acc{MIDI} 'kann' oder nicht: es hängt nur davon ab, ob sie selbst ein \acc{MIDI}-Plugin hat und ob es auch installiert ist.

Der Ausweg aus dem Dilemma ist einfach: Man macht den \acc{vlc} zum Standardtool für \acc{MIDI}-Dateien. Dazu klickt man mit der rechten Maustaste\footnote{bei Linkshändern die linke} auf eine \acc{MIDI}-Datei und wählt 'Eigenschaften' aus. In dem dann erscheinenden Dialog geht man zum Reiter \acc{Öffnen mit}, wählt aus der Liste möglicher Applikationen den \acc{vlc} aus und bestätigt die Wahl über den Button 'Set as Default'. Danach werden auch alle anderen \acc{MIDI}-Dateien automatisch mit dem  \acc{vlc} 'geöffnet' -- und das heißt in unserem Fall: abgespielt.

Will man \acc{MIDI}-Dateien dagegen über die Shell abspielen, benötigt man einen
\acc{MIDI}-Server mit Playerfunktionalität, will sagen: einen \enquote{software
synthesizer}.\footnote{$\rightarrow$
\href{https://wiki.ubuntuusers.de/TiMidity/}{https://wiki.ubuntuusers.de/TiMidity/}}
Dazu bietet sich \acc{timidity} an\footnote{$\rightarrow$
\href{http://timidity.sourceforge.net/{\#}info}{http://timidity.sourceforge.net/{\#}info}}
Auch in diesem Fall sollte man die besseren Soundbibliotheken installieren. Dann
muss man \acc{timidity} allerdings noch dazu 'überreden', diese auch zu nutzen:
man konfiguriert \acc{timidity} entsprechend, im dem man in der
Konfigurationsdatei \texttt{/etc/timidity/timidity.cfg} den Wert
\texttt{freepats.cfg} durch den Namen der Soundbibliothek, hier
\texttt{fluidr3\_gm.cfg} ersetzt. Ein entsprechendes Shellskript könnte so
aussehen:

\begin{verbatim}
#!/bin/bash
sudo apt-get install timidity
sudo apt install fluid-soundfont-gm fluid-soundfont-gs
# optional für ein Timidity eigenes Desktoptool
sudo apt-get install timidity-interfaces-extra
\end{verbatim}

Dann öffnet die Konfigurationsdatei mit \texttt{sudo vi /etc/timidity/timidity.cfg}, kommentiert die Zeile \texttt{source /etc/timidity/freepats.cfg} aus und löscht das Kommen\-tar\-zeichen vor \texttt{source /etc/timidity/fluidr3\_gm.cfg} respektive fügt diese Zeile ein.

Danach sollte man in einer Shell über das Kommando \texttt{timidity
MIDIDATEI.midi} seine \acc{MIDI}-Datei erfolreich abspielen können. Der Befehl
\texttt{timidity -ig MIDIDATEI.midi} öffnet sogar ein graphisches Interface.
Gelegentlich erwartet das eine oder andere Frontend, dass es seine Daten an
einen MIDI-Server übergeben kann. Auch dafür kann man \acc{timidity} benutzen.
Man startet ihn von der Shell oder im Bootprozess mit der Option {iA} und erhält
als Output eine Liste von Ports, auf denen \acc{timidity} als Server lauscht.
Konkret kann das so aussehen:
\begin{verbatim}
> timidity -iA
TiMidity starting in ALSA server mode
Opening sequencer port: 129:0 129:1 129:2 129:3
\end{verbatim}
Die Programme, die ihre Midi-Musidaten an einen solchen Server übergeben wollen,
muss man dann nur noch entsprechend konfigurieren.

Hat man all dies erledigt, kann man auch die Frontends der Notensatzsysteme zum
Anhören nutzen, die zwar generell \acc{MIDI}-bereit sind, die selbst aber keine
eigene oder nur eine unvollständige \acc{MIDI}-Umgebung mitbringen. Insbesondere
sollten nun auch \acc{elysium}, \acc{frescobalid} und \acc{rosegarden} als
\acc{MIDI}-'Player' funktionieren. Allerdings wird man diesen über ihr
jeweiliges Konfigurationsmenue\footnote{oft erreichbar über 'Edit/Preferences'}
meist noch mitteilen müssen, dass sie über den erzeugten \acc{Timidity}-Port
kommunizieren sollen.

Bliebe noch zu erwähnen, dass bestimmte Distributionen das
\acc{gstreamer}-System als Audio- und Videointerface
nutzen.\footnote{$\rightarrow$
\href{https://gstreamer.freedesktop.org/}{https://gstreamer.freedesktop.org/}}
Seine Fähigkeiten erwibt dieser Layer über Plugins. Hat man also kein
gstreamer-\acc{MIDI}-Plugin
 installiert, kann keines der Programme, die diesen Audiolayer nutzen,
 \acc{MIDI}-Dateien abspielen. Unglücklicherweise werden
 gstreamer-\acc{MIDI}-Plugins nur über die
 \acc{gstreamer-plugins-\textbf{bad}}-Serie angeboten, nicht über die
 \acc{gstreamer-plugins-\textbf{good}-Serie}.\footnote{$\rightarrow$
 \href{https://gstreamer.freedesktop.org/documentation/plugins.html}
 {https://gstreamer.freedesktop.org/documentation/plugins.html}} Mithin wird
 zumindest der Purist auch dann noch Programme finden, die \acc{MIDI}-Dateien
 nicht abspielen können, wenn er sein System auf der tieferen
 \acc{ALSA}/\acc{PulseAudio}-Ebene längst schon\acc{MIDI}-fähig gemacht
 hat.\footnote{Für tiefer gehende MIDI-Aspekt unter Linux \cite[vgl.][\nopage
 wp.]{Felix2018a}.}

% this is only inserted to eject fault messages in texlipse
% \bibliography{../bib/literature}
