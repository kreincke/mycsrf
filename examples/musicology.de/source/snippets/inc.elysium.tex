% mycsrf 'for beeing included' snippet template
%
% (c) Karsten Reincke, Frankfurt a.M. 2012, ff.
%
% This text is licensed under the Creative Commons Attribution 3.0 Germany
% License (http://creativecommons.org/licenses/by/3.0/de/): Feel free to share
% (to copy, distribute and transmit) or to remix (to adapt) it, if you respect
% how you must attribute the work in the manner specified by the author(s):
% \newline
% In an internet based reuse please link the reused parts to mycsrf.fodina.de
% and mention the original author Karsten Reincke in a suitable manner. In a
% paper-like reuse please insert a short hint to mycsrf.fodina.de and to the
% original author, Karsten Reincke, into your preface. For normal quotations
% please use the scientific standard to cite
%


%% use all entries of the bibliography

\subsubsection{Elysium ($\bigstar\bigstar\bigstar\bigstar$)}

\acc{Elysium} fungiert als Frontend für \acc{LilyPond}, ist aber kein
eigenständiger Editor, sondern wird zusammen mit \acc{Eclipse} genutzt:
\acc{Eclipse} ist eine integrierte Entwicklungsumgebung, deren Funktionalität
über die Integration von Plugins entsteht. Es wird von der \acc{Eclipse
Foundation} gepflegt, die allerdings weit mehr sein will, als der bloße
\acc{Eclipse}-Distributor.\footcite[vgl.][\nopage wp]{Eclipse2018a} Es gibt die
verschiedensten vorkonfigurierten Pakete für die unterschiedlichsten
(Programmier-)Zwecke, vorbereitet für die gängigen Betriebssysteme. In unserem
Fall reicht das einfache Standardpaket \enquote{Eclipse IDE for Java
Developers}.\footcite[vgl.][\nopage wp]{Eclipse2018b}

Ein Kandidat für die Erweiterung von deren Funktionalität wäre  in unserem
Kontext z.B. \acc{{\TeX}lipse}. Es macht \acc{Eclipse} zu einem exezellenten 'Editor'
für \LaTeX-Texte.\footcite[vgl.][\nopage wp]{TeXlipse2019a}

Das für uns entscheidene Plugin allerdings heißt \acc{Elysium} und wird aus
Eclipse heraus vom \enquote{Eclipse-Marketplace} heruntergeladen und
installiert.\footcite[vgl.][\nopage wp]{Harmath2019a} Sein Schöpfer -- Dénes
Harmath -- nennt seine Erweiterung die \enquote{LilyPond IDE für
Eclipse}.\footcite[vgl.][\nopage wp]{Harmath2019b} Der Name \acc{Elysium}
verweise auf \acc{Eclipse} und \acc{.ly}, der Extension von LilyPond-Dateien und
stehe für eine \enquote{himmlische} Verbindung: Schließlich sei beides
Open-Source-Software, wobei \enquote{[\ldots] writing complex scores with
LilyPond inevitably requires a more agile, more managed approach than a simple
command line and plain text editor}. Und eben das unterstütze \acc{Eclipse} als
bewährte Entwicklungsumgebung schon von sich aus.\footcite[vgl.][\nopage
wp]{Harmath2019d} Dem entsprechend ist \acc{Elysium} als freie Software
quelloffen unter der \acc{Eclipse Public License} publiziert
worden\footcite[vgl.][\nopage wp]{Harmath2018a}. Das System werde -- wie es
heißt -- in vier Schritten bereitgestellt: Sofern es die eigene Distribution
nicht schon mit sich bringe, installiere man zuerst auf die gewohnte Weise
\acc{LilyPond}, dann \acc{Eclipse} und von da aus das Plugin
\acc{Elysium}.\footcite[vgl.][\nopage wp]{Harmath2019c} Alle Varianten liefen
bei uns problemlos durch.

Von seinen Eigenschaften her ist \acc{Elysium} ein semi-graphischer Editor: man
gäbe -- Editor gestützt -- den gewünschten \acc{LilyPond}-Code ein und bei jeder
Sicherung der Quelldatei würden die entsprechende MIDI- und die zugrhörige
PDF-Datei kompiliert und angezeigt.\footcite[vgl.][\nopage wp]{Harmath2019e}
Unsere Referenzkadenz II kann \acc{Elysium} problemlos erfassen und sogar
hörbar machen:




% this is only inserted to eject fault messages in texlipse
%\bibliography{../bib/literature}
