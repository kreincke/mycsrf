% mycsrf 'for beeing included' snippet template
%
% (c) Karsten Reincke, Frankfurt a.M. 2012, ff.
%
% This text is licensed under the Creative Commons Attribution 3.0 Germany
% License (http://creativecommons.org/licenses/by/3.0/de/): Feel free to share
% (to copy, distribute and transmit) or to remix (to adapt) it, if you respect
% how you must attribute the work in the manner specified by the author(s):
% \newline
% In an internet based reuse please link the reused parts to mycsrf.fodina.de
% and mention the original author Karsten Reincke in a suitable manner. In a
% paper-like reuse please insert a short hint to mycsrf.fodina.de and to the
% original author, Karsten Reincke, into your preface. For normal quotations
% please use the scientific standard to cite
%


%% use all entries of the bibliography

\subsection{Canorus ($\bigstar\bigstar\bigstar$)}

\parpic(0.6cm,0.6cm)[r][t]{\includegraphics[width=0.5cm]{logos/canorus-300dpi.png}}
\label{Canorus}\acc{Canorus} ist ein Notensatzprogramm\footcite[vgl.][\nopage
wp.]{Canorus2019a}, das gelegentlich auch als Nachgolger von \acc{NoteEdit}
gehandelt wird.\footcite[vgl.][\nopage wp.]{WpedCanorus2019a} Die vorletzte
Version (0.7.2) aus dem Jahr 2015 wurde noch als \enquote{leichtgewichtige
Alternative} zu umfangreicheren Notensatzprogrammen
bezeichnet.\footcite[vgl.][\nopage wp.]{Kreussel2015a} Der letzte Releasekandidat
(0.7.3) stammt vom Juni 2018.\footnote{\cite[vgl.][\nopage wp.]{Canorus2019b}.
Dem Download beigelegt ist der Text der GPL-3.0. Das Programm wird also unter
einer anerkannten Open-Source-Lizenz distribuiert ($\rightarrow$
\href{https://opensource.org/licenses/alphabetical}
{https://opensource.org/licenses/alphabetical}). } Reviews der neueren Version
stehen noch aus.\footnote{Stand 01/2019} Vom Typ her gehört \acc{Canorus} zu den
graphischen Editoren: es erlaubt die Bearbeitung von Noten, ohne Text
'programmieren' zu müssen.

Die Homepage von \acc{Canorus} ist die
Sourceforge-Projektseite.\footcite[vgl.][\nopage wp.]{Canorus2019a} Aktuell wird
das Programm nicht in allen Distributionen angeboten\footnote{so nicht in Ubuntu
18.04}; externe Pakete gibt es eher für ältere
Programmsammlungen.\footcite[vgl.][\nopage wp.]{RepoCanorus2019a} Dies dürfte
einen einfachen Grund haben: Der -- von 2019 aus gesehene -- vorletzte
veröffentlichte Release-Candidat 0.7.2 stammte aus dem Jahr 2015, der letzte aus
2018.\footcite[vgl.][\nopage wp.]{Canorus2019b} Über drei Jahre tat sich -- von
außen gesehen - in Sachen Weiterentwicklung also wenig. In der Opensourcewelt
ist das üblicherweise ein Zeichen dafür, dass das Projekt 'eingeschlafen'
ist.\footnote{Auch vor 2015 soll es schon Unterbrechungen bei der Entwicklung
gegeben haben \cite[vgl.][\nopage wp.]{UbuntuCanorus2014a}} Die Veröffentlichung
vom Juni 2018 kam dann zu spät für z.B. die Ubuntu 18.04, der
LTS-Distribution.\footnote{Long-Term-Service-Distributionen erscheinen seltener
(bei Ubuntu alle 2 Jahre), werden dafür aber länger mit Updates versorgt.}
Gleichwohl sind wir sicher, dass das Programm in kommenden Distributionen wieder
aufgenommen wird.

So bleibt Anfang 2019 im Wesentlichen nur die Installation aus den Quellen,
falls man \acc{Canorus} nutzen will. Das sollte allerdings Programmierer und
Systemkenner nicht sehr herausfordern. Das Softwarepaket liefert eine
Readme-Datei mit, die für verschiedene Umgebungen die nötigen Befehle
auflistet.\footnote{Für Ubuntu 18.04 konnten wir verifizieren, dass die
Kompilation einfach durchläuft, wenn man die geforderten Zusatzpakete -- wie
beschrieben -- installiert.} Alternativ bleibt nur, auf die nächste
Distribution zu warten, die uns dieses Programm wieder ohne Eigenarbeit zur
Verfügung stellt.

Außer der Visualisierung am Bildschirm bietet \acc{Canorus} selbst keinen
Notensatz. Für die anderen Zwecke nutzt es \acc{LilyPond} als Backend. Oder
anders gesagt: \acc{Canorus} fungiert als graphisches Frontend für das textuell
arbeitende \acc{LilyPond}. Das, was man mit \acc{Canorus} erarbeitet, speichert
es in einem eigenen \acc{XML}-Format. Importieren kann man (gegenwärtig)
\acc{MusicXML}- und \acc{Midi}-Dateien, exportieren auch Graphiken und
\acc{LilyPond-}Dateien\footnote{Laut Homepage gäbe es auch einen Im- und/oder
Export für \acc{ABC}- oder \acc{MusiX\TeX}-Notate (\cite[vgl.][\nopage
wp.]{Canorus2019a}). Der von uns getestete Releasekandidat 0.7.3 bot diese Option
(noch) nicht (mehr).}.

Die Arbeitsteilung zwischen \acc{Canorus} und \acc{LilyPond} hat zwei
Konsequenzen: Zum einen reicht die Installation von \acc{Canorus} allein nicht
aus, um arbeitsfähig zu werden. Auch  \acc{LilyPond} muss systemisch
bereitgestellt werden. Und zum anderen ist das, was man am Bildschirm sieht,
letztlich nicht das, was man als druckfertige PDF-Datei erhält. Hier zunächst
unsere Referenzkadenz, wie Canorus sie als PDF generiert. Wie kaum anders zu
erwarten, ähnelt sie sehr der 'puren' \acc{LilyPond}-Ausgabe:


\parpic(6cm,2.4cm)[l][t]{\includegraphics[width=6cm]{frontends/canorus/cadenca2-pdf-300dpi.png}}

Allerdings konnte der 4$\rightarrow$3 Vorhalt mit einer \Halb\ gegen zwei
gebundene \Vier\ im Bass offensichtlich nicht umgesetzt werden. Die
Funktionssymbole sind dagegen erkennbar -- ganz wie bei der Kadenz-I in der
reinen \acc{LilyPond}-Version\footnote{$\rightarrow$ S.
\pageref{LilyPondKadenzI}} -- als 'Liedtext' integriert worden. Hier wie da
gilt: ohne unsere kleine Zusatzbibliothek\footnote{$\rightarrow$ S.
\pageref{LilyPondFuncTheory}} können die Symbole der Funktionstheorie in einem
\acc{LilyPond}-Code nur in grober Form genutzt werden.

Dieser Ausgabe steht graphisch eine leicht anders gestaltete Eingabe gegenüber:
\begin{center}
\includegraphics[width=0.9\textwidth]{frontends/canorus/cadenca2-canorus.png}
\end{center}

Man erkennt auf der linken Seite die Konzexte, die für Spezialeingaben zu
aktivieren sind. Der Modus zur Eingabe von Noten wird über das Menu aktiviert.
Oben in der Liste erscheinen die konkreten Ausformungen. Letztlich klickt man
mit der Maus dort in das Notensystem, wo die gewünschte Note erscheinen soll;
Vorzeichen werden zuvor mit \texttt{+} oder \texttt{-} aktiviert. \acc{Canorus}
ist also recht intuitiv zu bedienen. Darum fällt es nur bedingt ins Gewicht, dass
das Handbuch bei der manuelle Installation aus den Quellen heraus nicht mit
kompiliert wird.\footnote{\ldots und bis dato auch nicht im Netz zu finden ist}

Für die Eingabe von 'Liedtext' stellt \acc{Canorus} einen besonderen Modus
bereit (unten grün), den man ebenfalls im linken Randbereich aktiviert und dann
über Maus und Tastatur bedient.

Außer dem Textmodus möchte \acc{Canorus} noch einen Modus zur Harmonieanalyse
bereitstellen (oben, gelb). In diesem Modus kann man -- wie es unser Bild zeigt
-- in der oberen Programmleiste Funktionssymbole auswählen und unter bestimmte
Noten einfügen. Tatsächlich versucht \acc{Canorus} sogar, den entsprechenden
Akkord zu analysieren.

Damit verfolgt \acc{Canorus} einen vielversprechenden Ansatz, der über den
anderer Notensatzsysteme hinausgeht. Gleichwohl ist das Ergebnis aus drei
Gründen noch nicht produktiv nutzbar: Zum ersten werden \acc{Canorus} eigenen
Analysesymbole nicht mit gedruckt und exportiert. Zum zweiten löst die Nutzung
dieses Modus noch viele Pro\-gramm\-ab\-stürze aus. Und drittens kann man in diesem
Modus noch keine Funktionsparallelen, keine Gegenklänge und keine
Doppeldominaten ausdrücken.\footnote{Die angebotene Alternative der
Zwischendominante funktioniert nicht zuverlässig.} Darum ist dieses sehr
innovative Verfahren für den heutigen Musikwissenschaftler praktisch nicht
verwendbar.

Der \acc{LilyPond}-Code, den \acc{Canorus} exportiert, kann dagegen schon heute
gut weiterverarbeitet werden. Er ist -- sofern man auf die eigene und die
\acc{Canorus}-Harmonieanalyse verzichtet -- gut strukturiert, lesbar und
reproduzierbar\footnote{$\rightarrow$ S.\pageref{ExportVerifikation}}
korrekt.\footnote{Aus Platzgründen verzichten wir auf den erneuten Abdruck des
\acc{LilyPond}-Codes} Ohne eine textuelle Nachbearbeitung in Sachen
Harmonieanalyse wird er aber für den Musikwissenschaftler nur bedingt sinnvoll
sein.

So geben wir dem Programm 3 von 5 Sternen: In Maßen kann es jetzt bereits als
Frontend für \acc{LilyPond} verwendet werden, es ist vielversprechend, wird
aktuell gepflegt und erfüllt die Basisanforderungen. Als $\beta$-Version ist es
aber -- dem eigenen Anspruch gemäß -- von der Funktionalität und von der
Stabilität her ebenso noch begrenzt, wie vom Handling. Praktisch wird
\acc{Canorus} momentan allenfalls für den Musikwissenschaftler als
\acc{LilyPond}-Frontend in Frage kommen, der bereits mit \acc{Canorus} vertraut
ist.


% this is only inserted to eject fault messages in texlipse
%\bibliography{../bib/literature}
