% mycsrf 'for beeing included' snippet template
%
% (c) Karsten Reincke, Frankfurt a.M. 2012, ff.
%
% This text is licensed under the Creative Commons Attribution 3.0 Germany
% License (http://creativecommons.org/licenses/by/3.0/de/): Feel free to share
% (to copy, distribute and transmit) or to remix (to adapt) it, if you respect
% how you must attribute the work in the manner specified by the author(s):
% \newline
% In an internet based reuse please link the reused parts to mycsrf.fodina.de
% and mention the original author Karsten Reincke in a suitable manner. In a
% paper-like reuse please insert a short hint to mycsrf.fodina.de and to the
% original author, Karsten Reincke, into your preface. For normal quotations
% please use the scientific standard to cite
%


%% use all entries of the bibliography

\subsection{MuX2d ($\bigstar$)}

\label{MuX2d}\acc{MuX2d} beschreibt sich als \enquote{WYSIWYM(ean) editor for
MusiXTeX}, will sagen: als \enquote{macro package for typesetting music with
TeX}, das unter der GPL veröffentlicht werde.\footcite[vgl.][\nopage
wp.]{Mux2d2000a} Ein Open-Source-\acc{MusiX\TeX}-Frontend weckt in unserem
Kontext natürlich besonderes Interesse. Die letzte Version ist allerdings schon
vor fast 20 Jahren als Release 0.2.4 erschienen.\footnote{\cite[vgl.][\nopage
wp.]{Mux2d2000b}. Auch die Projekseite gibt an, dass \acc{MuX2d} unter der GPL-2.0
distributiert werde. Es ist also freie Software.} Und sie wurde damals noch als
\enquote{quite early} bezeichnet.\footcite[vgl.][\nopage wp.]{Mux2d2000a} Beides
lässt Schlimmes erahnen.

Und in der Tat liefern gängige Distributionen keine \acc{muX2d}-Pakete mehr. Die
zu Download angeboten Binärversion verweigert mit dem Hinweis den Dienst, es
könne die \acc{libqt.so.2} nicht finden. Und die Sourcecodeversion kann nicht
kompiliert werden, weil dabei eine veraltete Version der \acc{qt}-Bibliothek
gesucht wird.

Damit scheidet für Musikwissenschaftler auch \acc{muX2D} als Editor aus, so gern
wir ihn auch angeboten hätten. Man könnte dieses Programm wohl mit einigem
Aufwand wiederbeleben. Angesichts der frühen Entwicklungsstadiums und des Alters
mag sich das aber als nicht fruchtbringend herausstellen. Einen Stern ist uns
die bloße Möglichkeit jedoch -- wie immer -- wert.
% this is only inserted to eject fault messages in texlipse
%\bibliography{../bib/literature}
