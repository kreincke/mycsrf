% mycsrf 'for beeing included' snippet template
%
% (c) Karsten Reincke, Frankfurt a.M. 2012, ff.
%
% This text is licensed under the Creative Commons Attribution 3.0 Germany
% License (http://creativecommons.org/licenses/by/3.0/de/): Feel free to share
% (to copy, distribute and transmit) or to remix (to adapt) it, if you respect
% how you must attribute the work in the manner specified by the author(s):
% \newline
% In an internet based reuse please link the reused parts to mycsrf.fodina.de
% and mention the original author Karsten Reincke in a suitable manner. In a
% paper-like reuse please insert a short hint to mycsrf.fodina.de and to the
% original author, Karsten Reincke, into your preface. For normal quotations
% please use the scientific standard to cite
%


%% use all entries of the bibliography
\section{Graphiken: es geht auch manuell ($\bigstar\bigstar$)}
\label{IncludeGraphics}

Lässt man die bisher diskutierten Techniken Revue passieren, so fällt auf, dass
einige von ihnen die generelle Techniken verwenden, fertige Graphiken in den
\LaTeX-Code zu integrieren, anstatt Notentext über ein \LaTeX-Modul prozessierbar
zu machen: \textit{ABC}\footnote{$\rightarrow$ S. \pageref{AbcGraphics}}
verwendete diese 'kleine Mogelei' ebenso wie \textit{PMX}\footnote{$\rightarrow$
S. \pageref{PmxGraphics}} und \textit{LilyPond}.\footnote{$\rightarrow$ S.
\pageref{LilyPondGraphics}} \textit{LilyPond} selbst hat beschrieben, was man tun
muss, wenn man so vorgehen will:

\begin{quote}\textit{\enquote{Wenn Sie in ein Dokument Grafiken Ihres
Musiksatzes einfügen möchten, so können Sie genauso vorgehen, wie Sie andere
Grafiken einfügen würden: Die Bilder werden getrennt vom Dokument im PostScript-
oder PNG-Format erstellt und können dann in \LaTeX\ oder HTML eingefügt
werden.}\footcite[vgl.][20]{LilyPond2018e} }\end{quote}

Und dabei geht es um eine sehr einfache \LaTeX-Technik:

\begin{itemize}
  \item Als erstes muss man -- wie zu erwarten -- in der \LaTeX-Präambel ein
  spezielles \LaTeX-Paket aktivieren, und zwar mit dem Befehl
  \texttt{\textbackslash{usepackage}\{graphicx,color\}}.
  \item Danach braucht man nur noch an den Stellen, wo die Graphiken erscheinen
  sollen, den Befehl
  \texttt{\textbackslash{includegraphics}\{PATH-To-YOUR-PICTURE\}} einzugeben.
  Wichtig ist dabei, dass man die Extension der Graphik nicht an die Graphik
  anzuhängen braucht: liegen an der stelle verschiedene Typen derselben Graphik
  (PNG, EPS oder PDF), verwendet \LaTeX\ eines davon.\footnote{Dazu noch zwei
  kleine Hinweise: (a) Wenn die Notensatzprogramme Postscriptgraphiken
  exportieren, muss man das Ergebnis in der Regel noch in EPS-Dateien
  konvertieren, um sie git in eine \LaTeX-Dokument einbinden zu können. Unter
  Linux dient dazu das Kommando \texttt{ps2eps}. (b) Wenn verschiedene
  Versionen eines Bildes als Dateien mit gleichem Hauptnamen und divergierender
  Formatextension im Ordner liegen, sollte man sehr wohl die gewünschte
  Dateiendung angeben. Andernfalls sucht sich \LaTeX\ nämlich irgendeines
  heraus, was unangenehme Seiteneffekte haben kann, nach deren Ursache man
  dann länger sucht.}
\end{itemize}

Wer diesen Weg geht, muss drei Tücken im Auge behalten:
\begin{enumerate}
  \item Die Bildgröße muss händisch auf die Druckbreite ausgelegt werden, sei es über
  ein Graphikprogramm, sei es über Parametrisierung des Befehls
  \texttt{\textbackslash{includegraphics}}.
  \item Die Auflösung der Graphik muss groß genug angelegt werden. Die übliche
  Auflösung von Bildern im Internet (72 dpi / pixel) ist nicht druckadäquat, es
  sollten mindestens 300 dpi sein.
  \item Der Zeilenumbruch muss manuell überwacht werden: zu lange Bilder werden
  oft auf die nächste Seite verlegt und erzeugen so Leerraum.\footnote{Die
  Alternative wäre, die Graphik in eine floating Umgebung einzubetten. Dann
  könnte sie aber auf einer anderen, nicht zum Text passenden Seite erscheinen
  und müsste gesondert referenziert werden.}
\end{enumerate}



So bleibt den Musikwissenschaftlern zuletzt immer noch der Ausweg, eine
Notendatei mit irgendeinem externen Pogramm zu erstellen, in diese mit einem
beliebigen Graphikprogramm die Analysesymbole 'manuell' einzufügen und das
Ergebnis mit dieser Methode in den \LaTeX-Text einzubinden.

% this is only inserted to eject fault messages in texlipse
% \bibliography{../bib/literature}
