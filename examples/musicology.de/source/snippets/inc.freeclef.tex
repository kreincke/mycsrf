% mycsrf 'for beeing included' snippet template
%
% (c) Karsten Reincke, Frankfurt a.M. 2012, ff.
%
% This text is licensed under the Creative Commons Attribution 3.0 Germany
% License (http://creativecommons.org/licenses/by/3.0/de/): Feel free to share
% (to copy, distribute and transmit) or to remix (to adapt) it, if you respect
% how you must attribute the work in the manner specified by the author(s):
% \newline
% In an internet based reuse please link the reused parts to mycsrf.fodina.de
% and mention the original author Karsten Reincke in a suitable manner. In a
% paper-like reuse please insert a short hint to mycsrf.fodina.de and to the
% original author, Karsten Reincke, into your preface. For normal quotations
% please use the scientific standard to cite
%

\subsection{Free Clef ($\bigstar$)}

\parpic(1cm,1cm)[r][t]{\includegraphics[width=1cm]{logos/freeclef-700dpi.png}}
\label{FreeClef}\acc{Free Clef} bezeichnet sich selbst als \enquote{lightweight
notation editor}, der es seinen Nutzern erlaube, Musik zu schreiben und im
MusicXML-Format zu exportieren. Die letzte veröffentlichte Version stammt vom
Juni 2008, es handelt sich allerdings noch um eine
\acc{Beta}-Version.\footnote{\cite[vgl.][\nopage wp.]{FreeClef2008a}. Die Homepage
sagt, dass Programm stünde unter der GPL-2.0 Lizenz. Damit wäre es echte freie
Software.}

Moderne Distributionen bieten keine Binärpakete für \acc{Free Clef}
an.\footnote{Jedenfalls nicht Ubuntu 18.04.} Der Sourcecode kann von der
Projektseite heruntergeladen werden. Allerdings verlangt diese Software bei
einer Installation aus den Quellen heraus die Kompatibilitätsbibliothek
\acc{wxwidgets} in der Version 2.8. Aktuell stellen besagte Distributionen aber
nur die Version 3.x bereit, weil sie selbst ja bereits auf GNOME-3 und damit auf
GTK-3 basieren.

Damit scheidet \acc{Free Clef} als Notensatzsystem aktuell (noch) aus. Dass er
in den Quellen vorliegt und das GNU-automake/autoconf-System benutzt, macht eine
'Wiederbelebung' nicht unmöglich. Dies ist uns -- wieder einmal -- wenigstens
einen Stern wert.



% this is only inserted to eject fault messages in texlipse
%\bibliography{../bib/literature}
