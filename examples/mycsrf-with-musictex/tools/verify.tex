% mycsrf bitex- / bibliography data test
%
% (c) Karsten Reincke, Frankfurt a.M. 2010, 2011, ff.
%
% This text is licensed under the Creative Commons Attribution 3.0 Germany
% License (http://creativecommons.org/licenses/by/3.0/de/): Feel free to share
% (to copy, distribute and transmit) or to remix (to adapt) it, if you respect
% how you must attribute the work in the manner specified by the author(s):
% \newline
% In an internet based reuse please link the reused parts to mycsrf.fodina.de
% and mention the original author Karsten Reincke in a suitable manner. In a
% paper-like reuse please insert a short hint to mycsrf.fodina.de and to the
% original author, Karsten Reincke, into your preface. For normal quotations
% please use the scientific standard to cite
%

\documentclass[
  DIV=calc,
  BCOR=5mm,
  11pt,
  headings=small,
  oneside,
  abstract=true,
  toc=bib,
  english,ngerman]{scrartcl}

%%% (1) general configurations %%%
\usepackage[utf8]{inputenc}

%%% (2) language specific configurations %%%
\usepackage[]{a4,babel}
\selectlanguage{ngerman}

% package for improving the grey value and the line feed handling
\usepackage{microtype}

%language specific quoting signs
\usepackage{csquotes}

% jurabib configuration
\usepackage[see]{jurabib}
\bibliographystyle{jurabib}
% mycsrf German jurabib configuration include module file 
%
% (c) Karsten Reincke, Frankfurt a.M. 2012, ff.
%
% This file is licensed under the Creative Commons Attribution 3.0 Germany
% License (http://creativecommons.org/licenses/by/3.0/de/): 
% For details see teh file LICENSE in the top directory

% the first time cite with all data, later with shorttitle
\jurabibsetup{citefull=first}

%%% (1) author / editor list configuration
%\jurabibsetup{authorformat=and} % uses 'und' instead of 'u.'
% therefore define your own abbreviated conjunction: 
% an 'and before last author explicetly written conjunction

% for authors in citations
\renewcommand*{\jbbtasep}{\ u.\ } % bta = between two authors sep
\renewcommand*{\jbbfsasep}{,\ } % bfsa = between first and second author sep
\renewcommand*{\jbbstasep}{\ u.\ }% bsta = between second and third author sep
% for editors in citations
\renewcommand*{\jbbtesep}{\ u.\ } % bta = between two authors sep
\renewcommand*{\jbbfsesep}{,\ } % bfsa = between first and second author sep
\renewcommand*{\jbbstesep}{\ u.\ }% bsta = between second and third author sep

% for authors in literature list
\renewcommand*{\bibbtasep}{\ u.\ } % bta = between two authors sep
\renewcommand*{\bibbfsasep}{,\ } % bfsa = between first and second author sep
\renewcommand*{\bibbstasep}{\ u.\ }% bsta = between second and third author sep
% for editors  in literature list
\renewcommand*{\bibbtesep}{\ u.\ } % bte = between two editors sep
\renewcommand*{\bibbfsesep}{,\ } % bfse = between first and second editor sep
\renewcommand*{\bibbstesep}{\ u.\ }% bste = between second and third editor sep

% use: name, forname, forname lastname u. forname lastname
\jurabibsetup{authorformat=firstnotreversed}
\jurabibsetup{authorformat=italic}

%%% (2) title configuration
% in every case print the title, let it be seperated from the 
% author by a colon and use the slanted font
\jurabibsetup{titleformat={all,colonsep}}
%\renewcommand*{\jbtitlefont}{\textit}

%%% (3) seperators in bib data
% separate bibliographical hints and page hints by a comma
\jurabibsetup{commabeforerest}

%%% (4) specific configuration of bibdata in quotes / footnote
% use a.a.O if possible
\jurabibsetup{ibidem=strict}
% replace ugly a.a.O. by ders., a.a.O. resp. ders., ebda.
% but if there are more than one author or girl writers?
\AddTo\bibsgerman{
  \renewcommand*{\ibidemname}{Ds.,\ a.a.O.}
  \renewcommand*{\ibidemmidname}{ds.,\ a.a.O.}
}
\renewcommand*{\samepageibidemname}{Ds.,\ ebda.}
\renewcommand*{\samepageibidemmidname}{ds.,\ ebda.}

%%% (5) specific configuration of bibdata in bibliography
% ever an in: before journal and collection/book-titles 

\renewcommand*{\bibjtsep}{in:\ }
\renewcommand*{\bibbtsep}{in:\ }

% ever a colon after author names 
\renewcommand*{\bibansep}{:\ }
% ever a semi colon after the title 
\renewcommand*{\bibatsep}{;\ }
% ever a comma before date/year
\renewcommand*{\bibbdsep}{,\ }

% let jurabib insert the S. and p. information
% no S. necessary in bib-files and in cites/footcites
\jurabibsetup{pages=format}

% use a compressed literature-list using a small line indent
\jurabibsetup{bibformat=compress}
\setlength{\jbbibhang}{1em}

% which follows the design of the cites and offers comments
\jurabibsetup{biblikecite}

% print annotations into bibliography
\jurabibsetup{annote}
\renewcommand*{\jbannoteformat}[1]{{ \itshape #1 }}

%refine the prefix of url download
\AddTo\bibsgerman{\renewcommand*{\urldatecomment}{RDL: }}

% we want to have the year of articles in brackets
\renewcommand*{\bibaldelim}{(}
\renewcommand*{\bibardelim}{)}

%Umformatierung des Reihentitels und der Reihennummer
\DeclareRobustCommand{\numberandseries}[2]{%
\unskip\unskip%,
\space\bibsnfont{(=~#2}%
\ifthenelse{\equal{#1}{}}{)}{, [Bd./Nr.]~#1)}%
}%

%Umformatierung Referenzverweises
\usepackage{xpatch}
\AfterFile{dejbbib.ldf}{%
  \xapptocmd{\bibsgerman}{%
     \def\inname{\ifjboxford in:\else\ifjbchicago in:\else in:\fi\fi}%
    \def\incollinname{\ifjboxford in:\else\ifjbchicago in:\else in:\fi\fi}%
  }{}{}%
}

% the field printed before ISBN, ISSN or URL is the bibfield note
% Hence: If you insert into the field note the type of the literature
% [ Print | [FreeWeb | BibWeb] / [ PDF | HTML ] ] then you now
% get entries like:
% Print: ISBN ....
% BibWeb / PDF => http...
% That's nice for dealing with electronic sources correctly
\DeclareRobustCommand{\jbissn}[1]{\unskip:\space ISSN #1}%
\DeclareRobustCommand{\jbisbn}[1]{\unskip:\space ISBN #1}%

\DeclareRobustCommand{\biburlprefix}{$\Rightarrow$ }
\DeclareRobustCommand{\biburlsuffix}{}



% language specific hyphenation
%mycsrfk Hyphenation Include Module text
%
% (c) Karsten Reincke, Frankfurt a.M. 2012, ff.
%
% This file is licensed under the Creative Commons Attribution 3.0 Germany
% License (http://creativecommons.org/licenses/by/3.0/de/): 
% For details see teh file LICENSE in the top directory
%


\hyphenation{
adae-quate
Co-die-rung
Da-tei-namen 
ge-hos-tet
Hy-brid-system
Lily-Pond 
Mehr-stimmig-keit 
Musik-wissen-schaft-ler
No-ta-tions-system
}




%%% (3) layout page configuration %%%

% select the visible parts of a page
% S.31: { plain|empty|headings|myheadings }
%\pagestyle{myheadings}
\pagestyle{headings}

% select the wished style of page-numbering
% S.32: { arabic,roman,Roman,alph,Alph }
\pagenumbering{arabic}
\setcounter{page}{1}

% no indent for paragraphs
\setlength{\parindent}{0pt}
\setlength{\parskip}{1.2ex plus 0.2ex minus 0.2ex}

% package for improving the grey value and the line feed handling
\usepackage{microtype}

%%% (4) general package activation %%%
%\usepackage{utopia}
%\usepackage{courier}
%\usepackage{avant}
\usepackage[dvips]{epsfig}

% graphic
\usepackage{graphicx,color}
\usepackage{array}
\usepackage{shadow}
\usepackage{fancybox}

%- start(footnote-configuration)

\deffootnote[1.5em]{1.5em}{1.5em}{\textsuperscript{\thefootnotemark)\ }}

%- end(footnote-configuration)

% package for macking tables with broken lines
\usepackage{multirow}

%for using label as nameref
\usepackage{nameref}

%integrate nomenclature
% mycsrf  Deutsch Nomenclation Declaration Include Module 
%
% (c) Karsten Reincke, Frankfurt a.M. 2012, ff.
%
% This file is licensed under the Creative Commons Attribution 3.0 Germany
% License (http://creativecommons.org/licenses/by/3.0/de/): 
% For details see teh file LICENSE in the top directory

\usepackage[intoc]{nomencl}
\let\abbr\nomenclature
% Deutsche Überschrift
%\renewcommand{\nomname}{Abbreviations}
\renewcommand{\nomname}{Abkürzungen}

\setlength{\nomlabelwidth}{.20\hsize}
\renewcommand{\nomlabel}[1]{#1 \dotfill}
% reduce the line distance
\setlength{\nomitemsep}{-\parsep}
\makenomenclature


% Hyperlinks
\usepackage{hyperref}
\hypersetup{bookmarks=true,breaklinks=true,colorlinks=true,citecolor=blue,draft=false}

\begin{document}

%% use all entries of the bliography

%%-- start(titlepage)
\titlehead{Bibliographietest}
\subject{Zum Testen \itshape{neuer bibliographischer Angaben}}
\title{Kommentierte Bibliographie}
\subtitle{samt textweisem Anmerkungsapparat in Fußnoten}
\author{Verifikator% mycsrf License Include Module
%
% (c) Karsten Reincke, Frankfurt a.M. 2012, ff.
%
% This file is licensed under the Creative Commons Attribution 3.0 Germany
% License (http://creativecommons.org/licenses/by/3.0/de/): 
% For details see teh file LICENSE in the top directory
%

\footnote{Dieses Tutorial wird unter der \textit{Creative Commons Share
Alike}-Lizenz (\textit{CC BY-SA 4.0}) veröffentlicht: Sie dürfen das Material
also -- grob gesagt -- in jedwedem Format oder Medium vervielfältigen und
weiterverbreiten, es remixen, verändern und darauf aufbauen -- und zwar für
beliebige Zwecke, sogar kommerzielle --, sofern Sie angemessene Urheber- und
Rechteangaben machen, einen Link zur Lizenz beifügen, Ihr abgeleitetes Werk
unter derselben Lizenz verbreiten und angeben, ob und wo Sie das Original in
welcher Hichsicht geändert haben. Details dazu finden Sie unter $\Rightarrow$
\lnka{https://creativecommons.org/licenses/by-sa/4.0/deed.de}.
Das Recht, diese Arbeit im Rahmen des üblichen wissenschaftlichen Verfahrens zu
zitieren, bleibt davon unbenommen: es gibt keine Pflicht, das zitierende Werk
unter dieselbe Lizenz zu stellen. Die Bedingung der \textit{an\-ge\-mes\-se\-nen
Urheber- und Rechteangaben} erfüllen Sie, indem Sie in ihrem Werk an prominenter
Stelle den Text einfügen: $\langle$ {\itshape Abgeleitet von \texttt{mind your
Scholar Research Framework / musicology.de } \copyright\ K. Reincke CC BY-SA 4.0
($\rightarrow$
\lnka{https://github.com/kreincke/mycsrf/tree/master/examples/musicology.de)} }
$\rangle$ \newline {\tiny [ Weil wir selbst diese Arbeit von \textit{mycsrf}
abgeleitetet haben, müssen wir noch den Hinweis hinzufügen: {\itshape Format
abgeleitet von \texttt{mind your Scholar Research Framework} \copyright K.
Reincke CC BY 3.0 DE \lnka{http://fodina.de/mycsrf}) }]}}

}

\maketitle
%%-- end(titlepage)

\begin{abstract}
\noindent \itshape
Der Umgang mit Quellennachweisen muss sorgsam überprüft werden. Dazu dient
dieser Testtext: Ein neue bibliographischer Eintrag in die bibliographische
Datenbank wird fallweise ausgetestet: der initiale Eintrag, das direkte
Wiederholungszitat - zuerst seitengleich, dann seitendifferent, die
Autorwiederholung und die Kurztitelwiederaufnahme. Bei Sammlungen sollte auch
die Sammlungswiederholung ausgetestet werden. Und schließlich erscheint die
ganze kommentierte Bibliographie.
\end{abstract}
\footnotesize
%\tableofcontents
\normalsize

\section{Initialzitat \& Repitition per $\backslash$footcite aus \emph{jurabib}}

Hier erscheint das \enquote{ Initialzitat mit vollständiger bibliographischer
Angabe}\footcite[vgl. dazu:][123ff]{Kraemer2012a}, gefolgt von dem
\enquote{ seitengleichen Wiederholungszitat}\footcite[vgl.
dazu:][123ff]{Kraemer2012a}, und das wiederum gefolgt vom \enquote{ seitendifferenten
Wiederholungszitat}\footcite[vgl. dazu:][125f]{Kraemer2012a}.

Nach Möglichkeit sollte hier ein anderes Werk desselben Autors zitiert
werden\footcite[vgl. dazu:][321]{Kraemer2012a}.

Hier wird ein ganz anderes Werk zitiert\footcite[vgl.
dazu:][42]{Kraemer2009a}, um im Anschluss das \enquote{
Kurztitel-Wiederholungszitat'}\footcite[vgl. dazu:][123]{Kraemer2012a}
abzu testen.

\section{Testzitat $\backslash$cite in $\backslash$footnote}

Hier folgt - jeweils als Referenz über die Anweisung $\backslash$cite in
$\backslash$footnote - zunächst das ``seitendifferente
Wiederholungszitat''\footnote{\cite[vgl. dazu:][125]{Kraemer2012a}},gefolgt
von einem Fremdzitat\footnote{\cite[vgl. dazu:][42]{Kraemer2009a}} und das
wiederum gefolgt vom `Kurztitel-Wiederholungszitat''\footnote{\cite[vgl.
dazu:][125]{Kraemer2012a}}.

% insert the bibliographical data here
\bibliography{../bib/literature}

\end{document}
